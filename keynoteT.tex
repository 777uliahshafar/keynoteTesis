\documentclass[table]{beamer}
%\documentclass[handout,table]{beamer}

%\includeonlyframes{tesis}
\input{preamble}

%\setbeameroption{hide notes} % Only slides
%\setbeameroption{show only notes} % Only notes
\setbeameroption{show notes on second screen=right} % Both

\usepackage{beamer_digiPH}
\usepackage{comment}
\usepackage{lipsum}

\title{TESIS}
\subtitle{Preferensi Pengunjung terhadap Ruang Publik di Kawasan Tepi Laut Senggol Parepare}
\author[Shafar]{Muhammad Uliah Shafar\\ Pembimbing: Dr. Ars. Ir. Wijayanti, M.Eng\\\hspace{2.5cm} Prof. Dr. Ir. Atik Suprapti, M.T.}
\mode<presentation>{\keywords{Schlüsselwörter durch Komma getrennt}}

\date[00.05.2018]{Mei 2022}

\begin{document}

\begin{withoutheadline}
%\begin{frame}
%	% Schriftgröße verkleinern
%	\small
%	% Absatzabstand einstellen
%	\setlength{\parskip}{0.75\baselineskip}
%
%	Dies ist die eLectures-Präsentationsvorlage der Virtuellen PH.
%
%	Die kommenden Folien in \LaTeX\ können und \textbf{\emph{sollen} Sie nach Wunsch adaptieren}, wir bitten Sie aber, sich alle genau anzusehen und durchzulesen! Denn:
%
%	Auf jeder Folie finden Sie Anregungen, Hilfestellungen und Hinweise, sowohl für die inhaltlich-didaktische Vorbereitung als auch die Präsentation online. Zumindest die vorhergehende Titelfolie sollte bitte auf jeden Fall in dieser Form Verwendung finden (Wiedererkennungswert!)
%
%	Bitte senden Sie Ihre überarbeitete PDF-Datei \textbf{vor dem gemeinsamen Testtermin} direkt an Ihre/n Co-Moderator/in. Er/sie bespricht sie mit Ihnen und lädt sie für die eLecture im Raum hoch. Der Kontakt zur Co-Moderation wird aktiv von dieser hergestellt.
%
%	\begin{block}{Zur didaktischen Planung der Stunde:}
%		Eine eLecture-Stunde geht oft sehr schnell vorbei! Gehen Sie daher bitte von max. \SI{75}{\percent} der Zeit für Ihre Präsentation aus, der Rest ist meist durch Einführung und Vorstellung, Interaktion, Rückfragen und Abschluss belegt. Je interaktiver Sie die eLecture halten, desto besser: besprechen Sie sich mit Ihrer CoModeration!
%	\end{block}
%\end{frame}

{\setbeamercolor{background canvas}{bg=digiPH_orange!20}
\begin{frame}
	%\maketitle
	% \maketitle funktioniert auch im Article-Modus
	 \titlepage

\notec{Assalamualaikum wr. wb, salam sejahtera bagi kita semua, kepada dewan juri yang saya hormati perkenalkan nama sya mh. uliah shfr, pada kesempatan kali ini saya akan mempersentsikan hasil penelitian saya yang berjudul preferensi pengunjung terhadap ruang publik di kawasan tepi laut senggol parepare}
\end{frame}
}

\end{withoutheadline}

\miniframesoff

{
\usebackgroundtemplate{%
\tikz\node[opacity=0.3,inner sep=0] {\includegraphics[height=\paperheight,width=\paperwidth]{pe1.png}};}
\begin{frame}[label=inhalt]{Topik}
	\tableofcontents
\notec{Adapun yang menjadi topik pembahasan saya kali ini meliputi \scriptsize b1}
\end{frame}}
\miniframeson

\section{Pendahuluan}

{\cca
\begin{frame}[label=tesis]{Latar Belakang}

\begin{columns}[c]
    \begin{column}{.5\textwidth}
    \begin{figure}
        \centering
        \includegraphics[width=0.8\textwidth]{figures/ltr}
%        \caption{Block diagram of a 1st order system.}
    \end{figure}
    \end{column}
    \begin{column}{.5\textwidth}
    \begin{figure}
        \centering
        \includegraphics[width=0.8\textwidth]{figures/amn2}
%        \caption{Step response of a 1st order system.}
    \end{figure}
    \end{column}
\end{columns}
\vspace{4pt}
\only<1>{Kawasan tepi laut(waterfront) memiliki {\large karakteristik} dan {\large perhatian} khusus mengingat pentingnya air sebagai sumber kehidupan.}

\onslide<2->{Berdasarkan rencana tata ruang wilayah kota, Parepare menetapkan kawasan strategis kota yang mendasari pembangunan kawasan objek wisata. Menurut Amanda(2020), pengembangan kawasan objek wisata masih menjadi prioritas Pemerintah.}

\notec{Indonesia menjadi negara yang memiliki daerah-daerah dengan garis pantai terpanjang.\scriptsize b1}
\notec{dalam strategi pengembangan pariwisata di kota parepare oleh dinas olahraga pemuda dan pariwisata kota parepare \scriptsize a2}

%	\begin{block}{Zur Didaktik: Ablauf und Lernziele zu Beginn sind sehr wichtig!\\
%		\small Folgendes gilt aber auch für alle anderen Auflistungen:}
%		\end{block}

\end{frame}

\begin{frame}[label=tesis]{Latar Belakang}

\onslide<1->{Parepare menjadikan kawasan tepi laut senggol sebagai kawasan strategis kota yang memunculkan pengembangan dalam rangka pertumbuhan ekonomi.}

\only<2>{
\begin{columns}[c]
    \begin{column}{.5\textwidth}
    \begin{figure}
        \centering
        \includegraphics[width=0.8\textwidth]{figures/pkldl1}
%        \caption{Block diagram of a 1st order system.}
    \end{figure}
    \end{column}
    \begin{column}{.5\textwidth}
    \begin{figure}
        \centering
        \includegraphics[width=0.8\textwidth]{figures/pkldl2}
%        \caption{Step response of a 1st order system.}
    \end{figure}
    \end{column}
\end{columns}
}
\vspace{4pt}
\only<2>{\centering Tepi laut Senggol dulu}

\visible<3->{
\begin{columns}[c]
    \begin{column}{.5\textwidth}
    \begin{figure}
        \centering
        \includegraphics[width=0.8\textwidth]{figures/rs}
%        \caption{Block diagram of a 1st order system.}
    \end{figure}
    \end{column}
    \begin{column}{.5\textwidth}
    \begin{figure}
        \centering
        \includegraphics[width=0.8\textwidth]{figures/amn2}
%        \caption{Step response of a 1st order system.}
    \end{figure}
    \end{column}
\end{columns}
\vspace{4pt}
\onslide<3->{\centering Tepi laut Senggol sekarang}
}
\notec{Pada gambar kita lihat tepi laut senggol sebelum pengembangan. Terdapat sejumlah tempat pkl yang belum tertata dan masih kurang menarik.  \scriptsize a2}
\notec{Gambar ini merupakan tepi laut setelah pengembangan yaitu kondisi sekarang. Pengembangan ini cukup meningkatkan kondisi tepi laut sekarang dengan penambahan naungan, lantai, dan tempat makan.}

\notec{Menurut Kim (2012), ruang merupakan tempat yang dapat meningkatkan kualitas hidup dengan  memenuhi kebutuhan individu.
Pemenuhan kebutuhan individu dapat tercapai melalui pemahaman preferensi terhadap ruang yang lebih baik.\scriptsize a3}

\end{frame}}%end cca

\begin{frame}[label=tesis]{Rumusan Masalah}
\onslide<1->{Hussein (2014) menjelaskan, {\Large pengembangan tepi laut} yang berhasil mampu menarik masyarakat untuk {\Large datang ke pesisir.}}

\visible<2->{\centering\includegraphics[width=0.65\textwidth]{figures/org}}
\notec{Pada tahun 2011, kota Parepare memulai perencanaan pengembang­ an kawasan tepi laut senggol.\scriptsize a2}

\end{frame}

\begin{frame}[label=tesis]{Rumusan Masalah} \vspace{4pt}

\only<1>{
\centering    \includegraphics[width=0.85\textwidth]{figures/pe1}

Ruang A}

\only<2>{
\centering    \includegraphics[width=0.85\textwidth]{figures/pe2}

Ruang B}

\notec{Pengembangan ini membagi kawasan menjadi dua bagian ruang yang berbeda. \scriptsize a1}
\notec{ Kedua ruang ini memiliki perbedaan yang cukup signifikan dilihat dari pelaku utama yang menjajakan makanan, pengaturan elemen­elemen, sifat­sifat elemen ruang, dan lain­lain. Umumnya ruang A (seperti yang dilihat pada gambar) dikenal dengan tatanan dan elemen buatan yang mewah dan penuh elemen buatan \scriptsize a1}
\notec{sedangkan ruang B dikenal dengan tatanan yang terlihat lebih alami dengan sedikit elemen buatan. \scriptsize a2}

\end{frame}

\begin{frame}[label=tesis]{Rumusan Masalah} \vspace{4pt}

\begin{columns}[c]
    \begin{column}{.5\textwidth}
    \visible<1->{
    \begin{figure}
        \centering
        \includegraphics[width=0.95\textwidth]{figures/amn1}
%        \caption{Block diagram of a 1st order system.}
    \end{figure}}
    \end{column}
    \begin{column}{.5\textwidth}
    \only<2>{
    \begin{figure}
        \centering
        \includegraphics[width=0.95\textwidth]{figures/lmpub(2)}
%        \caption{Step response of a 1st order system.}
    \end{figure}}
    \end{column}
\end{columns}
\notec{Dengan demikian, pengembangan ini memunculkan dua kondisi yang kontras sehingga orang berkesempatan memilih ruang satu daripada lainnya.\scriptsize a1}

\notec{Dalam waktu tertentu, ketidakselarasan dapat terjadi pada kawasan tepi laut senggol dimana ruang satu terlihat lebih berhasil menarik pengunjung daripada yang lainnya.
Oleh karena itu, mengetahui preferensi pengunjung pada masing-masing ruang tersebut adalah penting (madureira,2018). \scriptsize a2}
\end{frame}

\begin{frame}[label=tesis]{Rumusan masalah} \vspace{4pt}
\begin{block}

\begin{center}
Menurut Mudureira (2018), pengetahuan terkait {\Large preferensi pengunjung terhadap ruang} akan menjelaskan {\Large kebutuhan pengunjung} dalam merancang ruang publik kedepan yang \textbf{efektif}.
\end{center}
\end{block}

\end{frame}



{\cca
\begin{frame}[label=tesis]{Pertanyaan Penelitian} \vspace{4pt}
\begin{blockcolor}

\begin{enumerate}
    \item<1-> Ruang apa yang dipilih dan bagaimana proporsi pengunjung terhadap ruang yang dipilih?
    \item<2-> Mengapa orang memiliki preferensi pada ruang tersebut?
    \item<3-> Apa saja elemen-elemen yang ada pada ruang publik sehingga pengunjung lebih memilih ruang tersebut?
\end{enumerate}

\end{blockcolor}

\notec{Berdasarkan permasalahan itu, penelitian ini menyelidiki preferensi  masyarakat terhadap ruang di kawasan tepi laut senggol. Maka penelitian ini menjawab sejumlah pertanyaan penelitian sebagai berikut: \scriptsize b1}
\end{frame}}
%------------------------------------------------------------------------------------

\section{Tinjauan Pustaka}
\begin{frame}[label=tin]{Definisi Umum} \vspace{4pt}
\begin{block}
\onslide<1->{\textbf{Ruang terbuka publik(rtp)}:
Ruang adalah {\large area} yang kontinu atau terbentang dan bersifat bebas atau tak terpakai (Oxford).}
\end{block}

\begin{block}
\onslide<2->{\textbf{Tepi laut}:
Tepi laut atau waterfront menurut KBBI adalah {\large wilayah pesisir.} Berdasarkan kamus Amerika Oxford menyebutkan tepi laut adalah "bagian dari kota yang berdampingan dengan sungai, pelabuhan atau danau." }
\end{block}

\begin{textblock*}{2cm}(10cm,7cm) % {block width} (coords.12,8cm)
\includegraphics[width=2cm]{figures/an-dic.png}
\end{textblock*}
\notec{adapun definisi umum penelitian ini adalah sebagai berikut:}

\end{frame}

\begin{frame}[label=tin]{Definisi Umum} \vspace{4pt}
\begin{block}
\onslide<1->{\textbf{Preferensi}:
Menurut Kamus Besar Bahasa Indonesia, preferensi adalah, 1 (hak untuk) didahulukan dan diutamakan daripada yang lain; {\large prioritas}; 2 pilihan; kecenderungan; kesukaan.}
\end{block}
\begin{block}
\onslide<2->{\textbf{Elemen Ruang}:
diartikan sebagai fisik ruang publik dimana mencakup kualitas, sifat, fitur dari sebuah ruang yang merupakan bagian dari tatanan ruang publik.}
\end{block}
\begin{textblock*}{2cm}(10cm,1.5cm) % {block width} (coords.12,8cm)
\includegraphics[width=2cm]{figures/an-dic.png}
\end{textblock*}
\end{frame}


{\cca
\begin{frame}[label=tin]{Variabel Penelitian: Aspek ruang}
\begin{exampleblock}{Aksesibilitas}
\onslide<1->{ Kemampuan individu untuk mendekati sesuatu (La Rosa et al. , 2018).
}
\end{exampleblock}
\begin{exampleblock}{Keamanan}
\onslide<2->{Bentuk perlindungan pada diri sendiri, keluarga, dan teman (Carr et al., 1992).
}
\end{exampleblock}
\begin{exampleblock}{Estetika}
\onslide<3->{ Kebutuhan akan rasa keindahan (Hradilova et al., 2013).}
\end{exampleblock}
\begin{exampleblock}{Fasilitas}
\onslide<4->{ Penggunaan untuk memfasilitas dan mendukung aktivitas (Carmona et al., 2003).
}
\end{exampleblock}
\notec{adapun variabel penelitian terkait aspek ruang meliputi:}

\end{frame}}
{\cca
\begin{frame}[label=tin]{Variabel penelitian: Elemen ruang \& pelaku}

\begin{center}
\begin{exampleblock}{Elemen ruang}

\onslide<1->{\textbf{Jumlah pohon}}
\onslide<2->{Bentuk pohon}
\onslide<3->{\Large Lebar jalan}
\onslide<4->{\Large\textbf{ Permukaan jalan}}
\onslide<4->{\small Warna bunga}
\onslide<5->{\Large\textbf{Jenis kursi}}
\onslide<5->{\Large Pencahayaan jalan}
\onslide<6->{\textbf{Orientasi elemen}}
\onslide<7->{\small Tempat wisata air}
\onslide<7->{Bangunan penunjang}
\onslide<7->{\large Elemen air}
\end{exampleblock}

%\raggedleft put this on the right
\onslide<8->{
\begin{minipage}{.40\textwidth}
\begin{exampleblock}{Pelaku:}
\onslide<8->{\Large\textbf{Gender}}
\onslide<9->{\Large Usia}
\onslide<10->{\textbf{Suku}}
\onslide<11->{\small Pendidikan}
\end{exampleblock}
\end{minipage}}
\end{center}
\notec{adapun variabel terkait elemen ruang meliputi \scriptsize a1}
\notec{adapun variabel terkait pelaku yaitu \scriptsize a2}
\end{frame}}

\begin{frame}[label=tin]{Kerangka Penelitian}
\scalebox{0.7}{
\begin{tikzpicture}[node distance=1em]

    \node (elm) [text width= .8\textwidth] {\centering\textbf{Variabel Penelitian}

            \begin{tabular}{p{.33\textwidth}p{1.2em} p{1.2em}p{1.2em}p{1.2em}p{1.2em}p{1.2em}p{1.2em}p{1.2em}p{1.2em}p{1.2em}p{1.2em}p{1.2em}p{1.2em}p{1.2em}}
\footnotesize
\bfseries\diagbox[innerleftsep=10pt,innerrightsep=3pt,width=9em, height=3.3cm]{Elemen\\Ruang}{Pelaku} &

{\rotatebox[origin=c]{90}{\parbox[c]{3cm}{\textbf{Gender}}}} & {\rotatebox[origin=c]{90}{\parbox[c]{3.3cm}{Laki-laki}}} & {\rotatebox[origin=c]{90}{\parbox[c]{3.3cm}{Perempuan}}} &

{\rotatebox[origin=c]{90}{\parbox[c]{3cm}{\textbf{Usia}}}} &
{\rotatebox[origin=c]{90}{\parbox[c]{3.3cm}{Remaja}}}&
{\rotatebox[origin=c]{90}{\parbox[c]{3.3cm}{Dewasa}}}&
{\rotatebox[origin=c]{90}{\parbox[c]{3.3cm}{Manula}}}&


{\rotatebox[origin=c]{90}{\parbox[c]{3cm}{\textbf{Suku}}}}&
{\rotatebox[origin=c]{90}{\parbox[c]{3.3cm}{Bugis}}}&
{\rotatebox[origin=c]{90}{\parbox[c]{3.3cm}{Non-bugis}}}&

{\rotatebox[origin=c]{90}{\parbox[c]{3cm}{\textbf{Pendidikan}}}}&
{\rotatebox[origin=c]{90}{\parbox[c]{3.3cm}{< SMA/sederajat}}}&
{\rotatebox[origin=c]{90}{\parbox[c]{3.3cm}{SMA/sederajat}}}&
{\rotatebox[origin=c]{90}{\parbox[c]{3.3cm}{Perguruan tinggi}}}\\
\toprule
\textbf{Aspek Ruang} &&&&&&&&&&&&&&\\
Aksesibilitas &&&&&&&&&&&&&&\\
Keamanan &&&&&&&&&&&&&&\\
\textit{dst...} &&&&&&&&&&&&&&\\
\textbf{Elemen Ruang} &&&&&&&&&&&&&&\\
Jumlah pohon &&&&&&&&&&&&&&\\
\tabitems \small{Sedikit pohon} &&&&&&&&&&&&&&\\
\tabitems \small{Bbrapa pohon} &&&&&&&&&&&&&&\\
Lebar jalan &&&&&&&&&&&&&&\\
\textit{dst...} &&&&&&&&&&&&&&\\
\end{tabular}};

\end{tikzpicture}
}
\notec{Kerangka penelitian yang digunakan adalah sebagai berikut \scriptsize b1}
\end{frame}

\section{Metode Penelitian}
{\ccl
\begin{frame}[label=met] \vspace{4pt}
\myft{Metode Penelitian}
\setbeamercolor{normal text}{fg=white}
\usebeamercolor[fg]{normal text}
\begin{columns}[t,onlytextwidth]
\fontsize{9pt}{10pt}\selectfont
\column{0.33\textwidth}\onslide<1->{
\textbf{Desain penelitian}
\begin{itemize}
    \item Bersifat kualitatif dan kuantitatif \textit{(cross-sectional)}
    \item Metode penelitian : Survei
\end{itemize}

\textbf{Metode Pengumpulan data}
\begin{itemize}
    \item Metode survei: Survei kualitatif dan kuantitatif
    \item Objek penelitian : Pengunjung tepi laut Senggol
\end{itemize}

}
\column{0.33\textwidth}\onslide<2->{
\textbf{Metode pengambilan sampel}
\begin{itemize}
    \item Teknik penyampelan adalah probabilitas dengan metode \textit{stratified random sampling}.
    \item Kriteria
    \item Jumlah sampel : 85 observasi
\end{itemize}
}
\column{0.33\textwidth}\onslide<3->{
\textbf{Metode Analisis Data}

\begin{itemize}
\item Analisis Deskriptif Analitik
\item Analisis Crosstabulasi
\item Analisis Biplot
\item[] \item Perangkat lunak : R untuk Crosstab dan Minitab untuk Biplot
\end{itemize}}
\end{columns}
\end{frame}}

\begin{frame}[label=obj]{Objek penelitian}
\centering
    \includegraphics[width=.8\textwidth]{figures/lokzi3.png}
    \notec{Objek penelitian ini terbagi atas dua bagian ruang yang disebut ruang A dan B. Luas ruang A meliputi daerah yang berwarna kuning. Sedangkan ruang B meliputi area yang berwarna merah. Ruang A terbentang dari pelabuhan nusantara Parepare hingga batas ruang B. Sedangkan ruang B terbentang dari batas ruang A hingga pasar senggol.}
\end{frame}

{\ccg
\begin{frame}[label=obj]
\centering
\begin{minipage}{.45\textwidth}
    \includegraphics[width=\linewidth]{figures/mapsra.jpg}
\end{minipage}
\begin{minipage}{.45\textwidth}
    \includegraphics[width=\linewidth]{figures/mapsrb.jpg}
\end{minipage}
\notec{Berikut ini view kedua ruang terlihat dari peta.}
\end{frame}


\begin{frame}[label=obj]
\centering
\only<1>{
    \includegraphics[width=.92\textwidth]{figures/ra}

    Ruang A
}
\only<2>{

    \includegraphics[width=.92\textwidth]{figures/rb}

    Ruang B
}
\notec{Selanjutnya view dari ruang A, dan ruang B.}
\end{frame}}

\section{Pembahasan}
\begin{frame}[label=hsl]{Gambaran responden} \vspace{2pt}

\begin{columns}[t,onlytextwidth]
\column{0.5\textwidth}
    \visible<1->{\includegraphics[width=.9\linewidth,trim= 1cm .4cm .5cm .8cm,clip]{figures/pieGender}\\}

   \visible<2->{\includegraphics[width=.9\linewidth,trim= 1cm .4cm .5cm .8cm,clip]{figures/pieUsia}\\}
\column{0.5\textwidth}
    \visible<3->{\includegraphics[width=.9\linewidth,trim= .8cm .5cm .35cm .6cm,clip]{figures/pieSuku}\\}

    \visible<4->{\includegraphics[width=.9\linewidth,trim= 1cm .3cm .5cm .55cm,clip]{figures/pieDidik}\\}
\end{columns}
\notec{Pada penelitian ini mayoritas responden adalah berjenis kelamin lali-laki dengan jumlah 58 orang.

35 dari mereka tergolong kelompok usia dewasa. Ini hampir setengah dari keseluruhan responden.

Mayoritas responden merupakan orang bugis, ini dikarenakan kota parepare kebanyakan adalah orang bugis. Dan sisanya adlaah orang makassar ataupun toraja.

Tingkat pendidikan terakhir responden kebanyakan adlaah perguruan tinggi dengan jumlah 43 orang. Ini melebihi stengah dari total responden.


}
\end{frame}

\begin{frame}[label=hsl]{Gambaran responden} \vspace{2pt}

\begin{columns}[t,onlytextwidth]

\column{0.5\textwidth}
    \visible<1->{\includegraphics[width=.9\linewidth,trim= .5cm .4cm .5cm .8cm,clip]{figures/piekerja}\\}

\column{0.5\textwidth}
\end{columns}
\notec{Mayoritas pekerjaan responden adalah karyawan sebanyak 29 orang.}
\end{frame}

{\cca
\begin{frame}[t,label=hsl2]{Ruang yang paling disukai} \vspace{4pt}

\begin{center}
    \includegraphics[width=.65\textwidth,trim= 1cm .3cm .5cm .6cm,clip]{figures/pieRuang}\\
\end{center}


\notec{Masuk pada pembahasan dalam menjawab pertanyaan penelitian. Dari keselurhan responden, mayoritas menyukai ruang A dengan jumlah 56 orang. Lainnya menyukai raung b sebanyak 29 orang.
}
\end{frame}}

\begin{frame}[t,label=hsl2]{Keragaman preferensi terhadap ruang}
\centering
\begin{minipage}{.55\textwidth}
\small
\begin{tabularx}{\linewidth}{ldd}
  \toprule
\multirow{2}{*}{Ruang} & \multicolumn{2}{c}{Gender} \\
 \cline{2-3} & \multicolumn{1}{c}{laki-laki } & \multicolumn{1}{c}{perempuan } \\
  \midrule
 ruang a  & 36 & 20 \\
  col \% & 62.1 & 74.1 \\
  \hline
 ruang b  & 22 &  7 \\
  col \% & 37.9 & 25.9 \\
   \hline
 \bottomrule
\end{tabularx}

\end{minipage}
\begin{minipage}{.55\textwidth}

\small
\begin{tabularx}{1\linewidth}{lddd}
  \toprule
\multirow{2}{*}{Ruang} & \multicolumn{3}{c}{Kelompok Usia} \\
 \cline{2-4} & \multicolumn{1}{c}{remaja} & \multicolumn{1}{c}{dewasa} & \multicolumn{1}{c}{manula} \\
  \midrule
 ruang a  & 25 & 20 & 11 \\
  col \% & 83.3 & 57.1 & 55.0 \\
  \hline
 ruang b  &  5 & 15 &  9 \\
  col \% & 16.7 & 42.9 & 45.0 \\
   \hline
 \bottomrule
\end{tabularx}

\end{minipage}
\notec{ Selanjutnya keseluruhan responden dibagi perkelompok berdasrkn gender, usia, suku dan pendidikan untuk mengetahui keragaman preferensi pengunjung terhadap ruang.}
\notec{Keragaman preferensi terhadap ruang berdasarkan gender, mayoritas laki-laki menyukai ruang a sebanyak 36 responden. Sedangkan mayoritas perempuan juga menyukai ruang a dengan jumlah 20 orang.}
\notec{Keragaman berdasarkan usia, kebanyakn usia remaja, dewasa, dan manula menyukai ruang A dengan jumlah 25, 20, dan 11 orang secara berurut. Ketiganya menunjukkan persentase diatas 50 persen.}
\end{frame}

\begin{frame}[t,label=hsl2]{Keragaman preferensi terhadap ruang}
\centering
\begin{minipage}{.55\textwidth}
\small
\begin{tabularx}{\linewidth}{ldd}
  \toprule
\multirow{2}{*}{Ruang} & \multicolumn{2}{c}{Suku} \\
 \cline{2-3} & \multicolumn{1}{c}{ bugis    } & \multicolumn{1}{c}{ nonbugis } \\
  \midrule
 ruang a  & 43 & 13 \\
  col \% & 71.7 & 52.0 \\
  \hline
 ruang b  & 17 & 12 \\
  col \% & 28.3 & 48.0 \\
   \hline
 \bottomrule
\end{tabularx}
\end{minipage}
\begin{minipage}{.99\textwidth}

\small
\begin{tabularx}{1\linewidth}{lddd}
  \toprule
\multirow{2}{*}{Ruang} & \multicolumn{3}{c}{Pendidikan} \\
 \cline{2-4} & \multicolumn{1}{c}{< sma / sederajat} & \multicolumn{1}{c}{sma / sederajat} & \multicolumn{1}{c}{perguruan tinggi} \\
  \midrule
 ruang a  &  8 & 23 & 25 \\
  col \% & 88.9 & 69.7 & 58.1 \\
  \hline
 ruang b  &  1 & 10 & 18 \\
  col \% & 11.1 & 30.3 & 41.9 \\
   \hline
 \bottomrule
\end{tabularx}
\end{minipage}
\notec{Keberagaman preferensi berdasarkan suku, mayoritas orang bugis menyukai ruang a sebanyak 43 orang. Sementara non bugis menunjukkan kesamaan prefernsi terhadp kedua ruang diantara 12-13 orng.  }

\notec{Kebragaman preferensi brdsrkan pendidikan, ketiga kelompok pendidikan, dibwh sma, sma dan perguruan tinggi menyukai ruang A. Meskipun demikian, perguruan tinggi menunjukkan preferesni terhadap ruang b lebih besar dari lainnya sebanyak 18 orang. }
\end{frame}

\begin{frame}[t,label=hsl2]{Keragaman preferensi terhadap ruang}
\centering

\footnotesize
\begin{tabularx}{1\linewidth}{lddddd}
  \toprule
\multirow{2}{*}{Ruang} & \multicolumn{5}{c}{Pekerjaan} \\
 \cline{2-6} & \multicolumn{1}{c}{ belum bekerja } & \multicolumn{1}{c}{ karyawan      } & \multicolumn{1}{c}{ lainnya       } & \multicolumn{1}{c}{ pelajar       } & \multicolumn{1}{c}{ wiraswasta    } \\
  \midrule
 ruang a  &  6 & 17 &  7 & 16 & 10 \\
  col \% & 100.0 & 58.6 & 58.3 & 76.2 & 58.8 \\
  \hline
 ruang b  &  0 & 12 &  5 &  5 &  7 \\
  col \% & 0.0 & 41.4 & 41.7 & 23.8 & 41.2 \\
   \hline
 \bottomrule
\end{tabularx}
\notec{Untuk yang terakhir, kebnyakan responden dari seluruh kelompok pekerjaan menyukai ruang A. Karyawan menunjukkan preferensi lebih besar terhadp ruang B dibanding kelompok pekerjaan lainnya. }
\end{frame}

{\cca
\begin{frame}[label=hsl3]{Preferensi terhadap aspek} \vspace{4pt}
%    \small
\scalebox{0.85}{
\begin{tabularx}{\textwidth}{r p{4.2em}
    p{4.2em}p{4.2em}p{4.2em}p{5em}}
    \hline
\bfseries\diagbox[innerleftsep=10pt,innerrightsep=3pt,width=7em, height=2.2cm]{Ruang}{Aspek\\Ruang}&
 {\rotatebox[origin=c]{90}{\parbox[c]{2.2cm}{\textbf{Aksesibili-\\tas}}}} & {\rotatebox[origin=c]{90}{\parbox[c]{2.2cm}{\textbf{Keamanan}}}} & {\rotatebox[origin=c]{90}{\parbox[c]{2.2cm}{\textbf{Estetika}}}} & {\rotatebox[origin=c]{90}{\parbox[c]{2.2cm}{\textbf{Fasilitas}}}} &{\rotatebox[origin=c]{90}{\parbox[c]{2.2cm}{\textbf{Total}}}} \\
 \toprule
Ruang A  & \ccta{32 (21\%)} & 5 (3\%)   & \ccta{37 (25\%)} & 30 (20\%) & 104 (69\%) \\
Ruang B  & 15 (10\%) & 2 (2\%)   & 9 (6\%) & \ccta{20 (13\%)} & 46 (31\%) \\
Total  & 47 (31\%) & 7 (5\%)  & 46 (31\%) & 50 (33\%) &150 (100\%) \\

 \bottomrule
\end{tabularx}}
\notec{Pada penelitian ini, mayoritas responden menyukai ruang A karena aspek estetika dan aksesibilitasnya. Sementara mayoritas responden menyukai ruang B karena aspek fasilitasnya. Secara keseluruhan, aspek fasilitas lebih disukai daripada aspek lainnya terhadap kedua ruang.}
\end{frame}}

\begin{frame}[label=hsl3]{Deskripsi tentang aspek} \vspace{4pt}

\onslide<1->{
\begin{exampleblock}

``Bisa menikmati laut lepas dan menghilangkan penat.."

\end{exampleblock}}

\onslide<2->{

\begin{block}

``Karena bersih, rapi dan memiliki pemandangan yang indah"

\end{block}}

\onslide<3->{
\begin{exampleblock}

``Lebih representatif dan nyaman tidak terkesan kumuh dan jorok.”

\end{exampleblock}}

\notec{Orang menilai estetika ruang A dengan beragam macam tanggapan, beberapa diantaranya sebagai berikut: }

\end{frame}

\begin{frame}[label=hsl3]{Deskripsi tentang aspek} \vspace{4pt}

\onslide<1->{
\begin{exampleblock}

``Tempatnya bagus untuk kumpul­kumpul dengan keluarga atau teman dll, pemandangan yang bagus."

\end{exampleblock}}

\onslide<2->{

\begin{block}

``Karena ruangan B adalah ruangan yang cocok untuk berlibur bersama keluarga dengan makan sambil duduk disamping pantai dengan keramaian"

\end{block}}
\notec{Pada ruang B, orang lebih mementingkan aspek fasilitas. Mereka menjelaskannya dengan beragam macam tanggapan, beberpa diantaranya sebagai berikut:}

\end{frame}


\begin{frame}[label=hsl3]{Keragaman preferensi diantara keseluruhan data}
\begin{figure}[htpb]
    \centering
    \includegraphics[width=.65\textwidth,trim={.4cm .3cm .4cm .1cm},clip]{figures/bra.png}
\end{figure}
\notec{Selanjutnya gambaran kebaragaman respon aspek dan ruang berdasarkn hasil crosstab dapat dilihat pada gambar biplot berikut ini.
Pada gambar ini aspek aksesibilitas berkontribusi signifikan terhadap kedua ruang. Itu dapat dijelaskan karena titik aksessibilitas terlihat sngat dekat dengan kedua vektor ruang.

Selanjutnya, aspek fasilitas lebih berkontribusi terhadap ruang B. Terlihat titik vektornya berhubungan dengan vektor ruang B.

Sedangkan, estetika mayoritas berkontribusi terhdap ruang A.

Untuk aspek keamanan memiliki pengaruh kecil terhadap ruang. Dengan jauhnya titik keaman terhadap kedua vektor ruang.

}
\end{frame}

\begin{withoutheadline}
\ccg{
\begin{frame}[label=hsl3]{Keragaman aspek diantara gender}
\begin{minipage}{.45\textwidth}
    \includegraphics[width=\linewidth]{figures/asplaki.png}
\end{minipage}
\begin{minipage}{.45\textwidth}
    \includegraphics[width=\linewidth]{figures/aspper.png}
\end{minipage}

\notec{ Selanjutnya keseluruhan responden dibagi perkelompok berdasrkn gender, usia, suku dan pendidikan untuk mengetahui keragaman preferensi pengunjung terhadap aspek.}

\notec{gambar biplot menunjukkan keragaman preferensi diantara kelompok gender hampir sama. Dimana preferensi terhadap ruang A sama-sama dipengaruhi oleh estetika. Sedangkan ruang B sama-sama dipengaruhi oleh fasilitas. Sementara kedua ruang sama-sama dipengaruhi oleh aspek aksesibilitas.}
%Gambar ini menunjukkan keragaman preferensi terhadap aspek ruang diantara gender.
%Pada presentasi ini, peneliti menyorot bahwa laki-laki dan perempuan mayoritas dipengaruhi oleh aspek aksesibilitas. Terlihat pada kedua gambar titik aksesibilitas sangat dekat dengan vektor kedua ruang.

\end{frame}}
\end{withoutheadline}

\begin{withoutheadline}
\ccg{
\begin{frame}[label=hsl3]{Keragaman aspek diantara Usia}
\begin{minipage}{.40\textwidth}
    \includegraphics[width=\linewidth]{figures/asprem.png}
\end{minipage}
\begin{minipage}{.40\textwidth}
    \includegraphics[width=\linewidth]{figures/aspdew.png}
\end{minipage}
\centering
\begin{minipage}{.40\textwidth}
    \includegraphics[width=\linewidth]{figures/aspman.png}
\end{minipage}

\notec{Sedangkan keragaman diantara usia, gambar biplot menunjukkan keragaman preferesi yang berbeda diantara remaja, dewasa dan manula.}
\notec{aspek aksesibilitas cukup berbeda diantara klp. gender. Untuk remaja aspek ini lebih berkontribusi terhadap ruang A, untuk dewasa aspek ini seimbang terhadap kedua ruang, brbeda dengan manula, aksesibilitas mayoritas berkontribusi terhadpa ruang B.}

%\notec{Gambar ini menunjukkan keragaman preferensi terhadap aspek ruang diantara usia.
%Pada presentasi ini, peneliti menyorot bahwa remaja mayoritas dipengaruhi oleh aksesibilitas, dewasa oleh estetika, dan manula oleh fasilitas.}


\end{frame}}
\end{withoutheadline}

\begin{withoutheadline}
\ccg{
\begin{frame}[label=hsl3]{Keragaman aspek diantara Suku}
\begin{minipage}{.45\textwidth}
    \includegraphics[width=\linewidth]{figures/aspbug.png}
\end{minipage}
\begin{minipage}{.45\textwidth}
    \includegraphics[width=\linewidth]{figures/aspnbug.png}
\end{minipage}

\notec{Sama dengan sebelumnya, keragaman diantara suku juga mengindikasikan adanya perbedaan.}
\notec{Aspek estetika dan fasilitas bersifat terbalik diantara klp suku. untuk suku bugis, estetika berkontribusi trhdp ruang A smntara fasilitas pada ruang B. sedngkan untuk suku non-bugis sebaliknya.}

%\notec{Sedangkan keragaman diantara usia, gambar biplot menunjukkan keragaman preferesi yang berbeda diantara remaja, dewasa dan manula.}

%\notec{Gambar ini menunjukkan keragaman preferensi terhadap aspek ruang diantara suku.
%Pada presentasi ini, peneliti menyorot bahwa bugis mayoritas dipengaruhi oleh fasilitas, dan nonbugis oleh estetika.}
\end{frame}}
\end{withoutheadline}

\begin{withoutheadline}
\ccg{
\begin{frame}[label=hsl3]{Keragaman aspek diantara Pendidikan}
\begin{minipage}{.40\textwidth}
    \includegraphics[width=\linewidth]{figures/aspbsma.png}
\end{minipage}
\begin{minipage}{.40\textwidth}
    \includegraphics[width=\linewidth]{figures/aspsma.png}
\end{minipage}
\centering
\begin{minipage}{.40\textwidth}
    \includegraphics[width=\linewidth]{figures/asppt.png}
\end{minipage}

\notec{begitu juga dengan keragaman diantara pendidikan  mengindikasikan adanya perbedaan.}
\notec{Aspek aksesibilitas mayoritas berkontribusi terhadap ruang A. dan tidak ada klp yang menunjukkn kontribusi pada ruang B.}


%\notec{Gambar ini menunjukkan keragaman preferensi terhadap aspek ruang diantara pendidikan.
%Pada presentasi ini, peneliti menyorot bahwa dibawah sma dan perguruan tinggi mayoritas dipengaruhi oleh estetika, sementara sma/sederajat oleh aksesibilitas.}
\end{frame}}
\end{withoutheadline}

{\cca
\begin{frame}[label=hsl4]
\footnotesize
\begin{tabular}[h!]{rP{5.2em}P{5.2em}P{5.4em} }
\hline
\bfseries\diagbox[width=8em, height=1.5cm]{Elemen \\ruang}{Kelompok}&{\bfseries\parbox[c][1.5cm]{\textwidth}{Ruang a}} &\textbf{Ruang b} & \textbf{Total}\\
\toprule

Sedikit pohon  & 6 (2\%)  & 3 (1\%)    &9(3\%) \\
Beberapa pohon  &10 (4\%)  & 1     &11 (4\%) \\

Cukup rindang  &2 (1\%)  &4 (1\%)   &6(2\%) \\
Sangat rindang  &28 (10\%)  & 6 (2\%)    &34(13\%) \\

1-3m  &3 (1\%)  & 6 (2\%)   &9(3\%) \\
> 3m  &\ccta{35 (13\%)}  & 7 (3\%)    &42(16\%)\\

Paving  &7 (3\%)  & 4 (1\%)    &11(4\%) \\
Aspal  &12(4\%)  & 1    &13(5\%) \\
Keramik  &12(4\%)  & 1    &13(5\%) \\

kursi bergerak  &6(2\%)  & 0    &6(2\%)\\
kursi dinding  &3(1\%)  & 2(1\%)    &5(2\%) \\

pencahayaan cukup   &5 (2\%)  &11(4\%)     &16(6\%) \\
pencahayaan tinggi   &\ccta{31 (12\%)}  & 6 (2\%)    &37(14\%)\\

membelakangi laut   &8(3\%)  & 3 (1\%)    &11(4\%)\\
menghadap laut   &29 (11\%)  & \ccta{15 (6\%)}    &44(16\%) \\

Total  & 197 (74\%)  & 70 (26\%) & 267 (100\%)   \\
 \bottomrule
\end{tabular}

\notec{Berdasarkan 267 total respon terkait elemen yang disukai, respon digolongkn berdsrkan mcam-mcam ruang yang terdiri dari ruang A dan B. Dari 23 elemen pada variabel penelitian, hnya 15 yang memiliki nilai signifikan untuk ditampilkn pada tabel.
Elemen yang paling disukai pada ruang A adlah lebar jalan lebih dari 3 meter serta penchyaan tinggi. Sedangkan pada ruang B, elemen yang disukai adalah kebnyakan orientasi yang menghadap ke laut. }

\end{frame}
}

{\cca
\begin{frame}[label=hsl4]{Preferensi terhadap kelompok elemen}
\begin{tabular}[ht]{rP{5em}P{5em}P{5.5em} }
\hline
\bfseries\diagbox[innerleftsep=8pt,innerrightsep=3pt,width=10em, height=2.5cm]{Elemen \\ruang}{Kelompok}&{\bfseries\parbox[c][2.5cm]{\textwidth}{Ruang a}} &\textbf{Ruang b} & \textbf{Total}\\


\toprule

Jumlah pohon  & 16 (6\%)  & 4 (1\%) & 20 (7\%)    \\
Bentuk pohon  & 30 (11\%)  & 10 (4\%) & 40 (15\%)   \\
Lebar jalan  & \ccta{38 (14\%)}  & 13 (5\%) &51 (19\%)   \\
Permukaan jalan  & 31 (12\%)  & 6 (2\%) & 37 (14\%)   \\
Jenis kursi  & 9 (3\%)  & 2 (1\%)  & 11 (4\%)  \\
Tingkat cahaya  & 36 (13\%)  & 17 (6\%) & 53 (20\%)   \\
Orientasi elemen  & 37 (14\%)  & \ccta{18 (7\%)} & 55 (21\%)   \\

Total  & 197 (74\%)  & 70 (26\%) & 267 (100\%)   \\

\bottomrule

\end{tabular}
\notec{Selanjutnya sifat-sifat elemen tersebut dikelompokkan menjadi 7 jenis elemen untuk mempermudah  analisis multivariate dalam mengetahui keragamannya. Berdasarkn ketujuh jenis elemen, mayoritas pengunjung di ruang A memiliki preferensi terhadp jenis elemen lebar jalan, sedangkan di ruang B menyukai orientasi elemen.}

\end{frame}}


\begin{frame}[label=hsl4]{Keragaman preferensi terhadap klp elemen}
\begin{figure}[htpb]
    \centering
    \includegraphics[width=.65\textwidth,trim={.5cm .3cm .5cm .1cm},clip]{figures/bre.png}
\end{figure}
\notec{Gambar ini menunjukkan bahwa elemen seprti bentuk pohon, lebr jlaan dan permukaan jalan berkontribusi terhadap ruang A. Terlihat pada gambar titik lebar jalan dan permukaan jalan berhubungan dengan vektor ruang A.
Elemen seperti orientasi elemen, dan tingkat cahaya mayoritas berkontribusi pada ruang B. Sama dengan sebelumnya, titik tingkat cahaya berhubungan dengan proyeksi tegak lurus ke vektor. }
\end{frame}

\begin{withoutheadline}
\ccg{
\begin{frame}[label=hsl4]{Keragaman elemen diantara Gender}
\begin{minipage}{.45\textwidth}
    \includegraphics[width=\linewidth]{figures/elelaki.png}
\end{minipage}
\begin{minipage}{.45\textwidth}
    \includegraphics[width=\linewidth]{figures/eleper.png}
\end{minipage}

\notec{Sama dengan keragaman preferensi terhadap aspek ruang, keseluruhan data preferensi terhadap elemen juga dikelompokkan berdasarkn latar belakng responden.}
\notec{Hasilnya menunjukkan adanya perbedaan keragaman diantara kelompok gender.}
\notec{Untuk laki-laki, bentuk phn dan lbr jlan mayoritas berkontribusi pd ruang A, dan tingkat cahaya berkontribusi pada ruang B. Untuk perempuan, tingkat chya lebih berkontribusi pd ruang A sedangkn orientasi berkontribusi pd ruang B.}

%\notec{Gambar ini menunjukkan keragaman preferensi terhadap kelompok elemen ruang diantara gender.
%mayoritas laki-laki dipengaruhi oleh lebar jalan yang sangat kuat pada ruang A. sedangkan kebanyakan perempuan dipengaruhi oleh tingkat cahaya khususnya di ruang A. }


\end{frame}}
\end{withoutheadline}

\begin{withoutheadline}
\ccg{
\begin{frame}[label=hsl4]{Keragaman elemen diantara Usia}
\begin{minipage}{.40\textwidth}
    \includegraphics[width=\linewidth]{figures/elerem.png}
\end{minipage}
\begin{minipage}{.40\textwidth}
    \includegraphics[width=\linewidth]{figures/eledew.png}
\end{minipage}
\centering
\begin{minipage}{.40\textwidth}
    \includegraphics[width=\linewidth]{figures/eleman.png}
\end{minipage}

\notec{keragmaan diantara kelompok usia juga berbeda}
\notec{elemen permukaan jalan memiliki kontribusi besar terhadap ruang A di smw klp.}
%\notec{Gambar ini menunjukkan keragaman preferensi terhadap elemen ruang diantara usia.
%Pada presentasi ini, peneliti menyorot bahwa remaja mayoritas dipengaruhi oleh lebar jalan, mayoritas dewasa dan manula tingkat cahaya.}
\end{frame}}
\end{withoutheadline}

\begin{withoutheadline}
\ccg{
\begin{frame}[label=hsl4]{Keragaman elemen diantara Suku}
\begin{minipage}{.45\textwidth}
    \includegraphics[width=\linewidth]{figures/elebug.png}
\end{minipage}
\begin{minipage}{.45\textwidth}
    \includegraphics[width=\linewidth]{figures/elenbug.png}
\end{minipage}
\notec{Sama halnya dengan keragaman diantara suku. }
\notec{Korelasi ruang diantara kedua klp ini sangat berbeda, bugis menunjukkan korelasi yang tinggi semntara korelasi nonbugis cukup kecil.  }
%\notec{Gambar ini menunjukkan keragaman preferensi terhadap elemen ruang diantara suku.
%Pada presentasi ini, peneliti menyorot bahwa mayoritas bugis dipengaruhi oleh tingkat cahaya dan orientasi elemen, dan nonbugis oleh bentuk pohon.}
\end{frame}}
\end{withoutheadline}

\begin{withoutheadline}
\ccg{
\begin{frame}[label=hsl4]{Keragaman elemen diantara Pendidikan}
\begin{minipage}{.40\textwidth}
    \includegraphics[width=\linewidth]{figures/elebsma.png}
\end{minipage}
\begin{minipage}{.40\textwidth}
    \includegraphics[width=\linewidth]{figures/elesma.png}
\end{minipage}
\centering
\begin{minipage}{.40\textwidth}
    \includegraphics[width=\linewidth]{figures/elept.png}
\end{minipage}
\notec{beigtu juga dengan diantara kelompok pendidikan terakhir, dimana terdapat perbedaan terhadap keragaman preferensi terhadap elemen ruang.}
\notec{permukaan jalan berkontribusi besar trhadap ruang A untuk smw klp pendidikan. Semtnara orientasi elemen berkontribusi besar pada ruang B kecuali untuk pendidikan dibawah sma.}

%\notec{Gambar ini menunjukkan keragaman preferensi terhadap elemen ruang diantara pendidikan.
%Pada presentasi ini, peneliti menyorot bahwa dibawah sma dipengaruhi oleh lebar jalan dan orientasi elemen. sma dipengaruhi oleh lebar jalan saja, dan perguruan tinggi oleh orientasi elemen.}

\end{frame}}
\end{withoutheadline}


{\cco
\begin{frame}[label=kes] \vspace{4pt}
\myft{Kesimpulan}
\setbeamercolor{normal text}{fg=white}
\usebeamercolor[fg]{normal text}
Setelah melakukan analisis, sejumlah pertanyaan penelitian terjawab yang kemudian disimpulkan sebagai berikut:

\begin{itemize}
    \item<1-> Hasil dari studi ini mengindikasikan ruang A paling disukai dibandingkan ruang B di tepi laut Senggol.
    \item<2-> Alasan yang mendasari orang memiliki preferensi pada ruang tertentu adalah aspek fasilitas. Kebanyakan orang menyukai aspek ini. Oleh karena itu, aspek fasilitas seharusnya diutamakan dalam desain dan pengelolaan ruang terbuka tepi laut.
    \item<3-> Elemen yang paling disukai oleh pengunjung adalah lebar jalan dalam memilih ruang tertentu. Pernyataan ini mendukung aspek aksesibilitas yang merupakan aspek yang disukai kedua.
\end{itemize}


\end{frame}}

%-------------------------------------------------------------------------------------
\begin{comment}
    \input{keynoteco.tex}
\end{comment}
{
\usebackgroundtemplate{%
\tikz\node[opacity=0.3,inner sep=0] {\includegraphics[height=\paperheight,width=\paperwidth]{pe1.png}};}
\begin{frame}<beamer>{}
\bibliographystyle{apalike}
{\tiny
\bibliography{biblio.bib}
}
\centering
\Huge{Terima kasih}
\notec{Sekian presentasi dari saya, lebih kurangnya mohon maaf. terima kasih atas perhatiannya.}

\end{frame}}

\end{document}
