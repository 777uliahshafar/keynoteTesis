\documentclass{beamer}

\includeonlyframes{current}
\input{preamble}

\setbeameroption{hide notes} % Only slides
%\setbeameroption{show only notes} % Only notes
%\setbeameroption{show notes on second screen=right} % Both

\usepackage{beamer_digiPH}
\usepackage{comment}
\usepackage{lipsum}

\title{TESIS}
\subtitle{Preferensi Pengunjung terhadap Ruang di Kawasan Waterfront Senggol Parepare}
\author{Muhammad Uliah Shafar}
\institute{21020119420029\\ \url{777uliahshafar@gmail.com}}

\mode<presentation>{\keywords{Schlüsselwörter durch Komma getrennt}}
\date[28.02.2022]{28 Februari 2022}

\begin{document}

\begin{withoutheadline}
%\begin{frame}
%	% Schriftgröße verkleinern
%	\small
%	% Absatzabstand einstellen
%	\setlength{\parskip}{0.75\baselineskip}
%
%	Dies ist die eLectures-Präsentationsvorlage der Virtuellen PH.
%
%	Die kommenden Folien in \LaTeX\ können und \textbf{\emph{sollen} Sie nach Wunsch adaptieren}, wir bitten Sie aber, sich alle genau anzusehen und durchzulesen! Denn:
%
%	Auf jeder Folie finden Sie Anregungen, Hilfestellungen und Hinweise, sowohl für die inhaltlich-didaktische Vorbereitung als auch die Präsentation online. Zumindest die vorhergehende Titelfolie sollte bitte auf jeden Fall in dieser Form Verwendung finden (Wiedererkennungswert!)
%
%	Bitte senden Sie Ihre überarbeitete PDF-Datei \textbf{vor dem gemeinsamen Testtermin} direkt an Ihre/n Co-Moderator/in. Er/sie bespricht sie mit Ihnen und lädt sie für die eLecture im Raum hoch. Der Kontakt zur Co-Moderation wird aktiv von dieser hergestellt.
%
%	\begin{block}{Zur didaktischen Planung der Stunde:}
%		Eine eLecture-Stunde geht oft sehr schnell vorbei! Gehen Sie daher bitte von max. \SI{75}{\percent} der Zeit für Ihre Präsentation aus, der Rest ist meist durch Einführung und Vorstellung, Interaktion, Rückfragen und Abschluss belegt. Je interaktiver Sie die eLecture halten, desto besser: besprechen Sie sich mit Ihrer CoModeration!
%	\end{block}
%\end{frame}

{\setbeamercolor{background canvas}{bg=digiPH_orange!20}
\begin{frame}
	%\maketitle
	% \maketitle funktioniert auch im Article-Modus
	 \titlepage
\end{frame}
}

\end{withoutheadline}

\miniframesoff
\begin{frame}[label=inhalt]{Topik}
	\tableofcontents
\end{frame}
\miniframeson

\section{Pendahuluan}

{\cca
\begin{frame}[label=tesis]{Latar Belakang}
Kawasan waterfront memiliki karakteristik dan perhatian khusus mengingat pentingnya air sebagai sumber kehidupan.

Munculnya tepi laut merupakan permintaan masyarakat terhadap akses ke tepi laut.

Menurut \cite{madureira2018}, tepi laut mampu meningkatkan kualitas hidup masyarakat dengan cara memenuhi kebutuhannya.

    \includegraphics[width=0.4\textwidth]{figures/ltr}\\

%	\begin{block}{Zur Didaktik: Ablauf und Lernziele zu Beginn sind sehr wichtig!\\
%		\small Folgendes gilt aber auch für alle anderen Auflistungen:}
%		\end{block}

\notec{}

\end{frame}}

\begin{frame}[label=tesis]{Latar Belakang}

Tepi laut Senggol merupakan kawasan pesisir laut di Parepare.

Belakangan ini, Parepare berfokus pada peningkatan sektor kepariwisataan.

Oleh karena itu, tepi laut Senggol sebagai salah satu tempat yang mendapat perhatian untuk pengembangan.
\end{frame}

{\cca
\begin{frame}[label=intro]
\myft{Rumusan masalah}
Setelah mengalami penataan, tepi laut Senggol terbagi atas dua macam ruang yang berbeda.

Munculnya perbedaan ruang ini, membuat pengunjung memilih ruang satu daripada lainnya. Sehingga salah satu ruang memungkinkan kurang terpilih yang lambat laun akan terbengkalai.

Agar kedua ruang tersebut memiliki potensi yang sama untuk dipilih, maka mengetahui kebutuhan masyarakat terhadap ruang adalah sangat penting.

Pemahaman terkait preferensi orang terhadap ruang akan menjelaskan kebutuhan masyarakat \citep{madureira2018}
\end{frame}}

{\cca
\begin{frame}[label=per]{Pertanyaan Penelitian} \vspace{4pt}
\begin{enumerate}
    \item Ruang apa yang dipilih pengunjung untuk dimanfaatkan sebagai ruang publik?
    \item Mengapa orang memiliki preferensi pada ruang tersebut?
    \item Apa saja elemen-elemen yang ada pada ruang publik sehingga pengunjung lebih memilih ruang tersebut?
\end{enumerate}
\notec{
(b) Berdasarkan permasalahan itu, penelitian ini menyelidiki preferensi  masyarakat terhadap ruang di kawasan tepi laut senggol. Maka penelitian ini menjawab sejumlah pertanyaan penelitian sebagai berikut:}
\end{frame}}
%------------------------------------------------------------------------------------

\section{Tinjauan Pustaka}
{\cca
\begin{frame}[label=tin]{Definisi} \vspace{4pt}
Tepi laut atau waterfront menurut KBBI adalah {\Large wilayah pesisir.} Berdasarkan kamus Amerika Oxford menyebutkan tepi laut adalah "bagian dari kota yang berdampingan dengan sungai, pelabuhan atau danau."\newline

Menurut Kamus Besar Bahasa Indonesia, preferensi adalah, 1 (hak untuk) didahulukan dan diutamakan daripada yang lain; {\Large prioritas}; 2 pilihan; kecenderungan; kesukaan.

Elemen ruang diartikan sebagai fisik ruang publik dimana mencakup kualitas, sifat, fitur dari sebuah ruang yang merupakan bagian dari tatanan ruang publik.
\end{frame}}

\begin{frame}[label=tin]{Instrumen} \vspace{4pt}
\begin{tikzpicture}[node distance=1cm]

    \node (ask) [startstop, text width= 7cm] {\textbf{ASPEK RUANG}\\

\begin{tabular}{ll}
    \tabitem Aksesibilitas & \tabitem Estetika \\
    \tabitem Keamanan & \tabitem Fasilitas\\
\end{tabular}};

    \node (elm) [startstop, below of=ask, yshift= -3cm, text width= 7cm] {\textbf{ELEMEN RUANG}

            \begin{tabular}{ll}
    \tabitem Jumlah pohon & \tabitem Warna bunga \\
    \tabitem Bentuk pohon & \tabitem Jenis kursi\\
    \tabitem Lebar jalan & \tabitem Pencahayaan jalan\\
    \tabitem Permukaan jalan & \\
\end{tabular}};

    \node (jun) [startstop, xshift=4cm, yshift= -1.8cm, text width= 4.5cm] {Menghasilkan temuan preferensi pada ruang publik tepi laut};

\draw [doublearrow] (ask) -- (elm);
\draw [arrow] (ask) -| (jun);
\draw [arrow] (elm) -| (jun);


\end{tikzpicture}

\end{frame}

\section{Metode Penelitian}
{\cca
\begin{frame}[label=met]{Metode Penelitian} \vspace{4pt}
\begin{columns}[t,onlytextwidth]
\fontsize{9pt}{10pt}\selectfont
\column{0.33\textwidth}
\textbf{Desain penelitian}
\begin{itemize}
    \item Bersifat kualitatif dan kuantitatif
    \item Metode penelitian : Survei
\end{itemize}

\textbf{Metode Pengumpulan data}
\begin{itemize}
    \item Metode survei: Survei kualitatif dan kuantitatif
    \item Objek penelitian : Pengunjung tepi laut Senggol
\end{itemize}


\column{0.33\textwidth}
\textbf{Metode pengambilan sampel}
\begin{itemize}
    \item Teknik penyampelan adalah nonprobabilitas atau \textit{purposive sampling}
    \item Kriteria
    \item Jumlah sampel : 85 observasi
\end{itemize}
\column{0.33\textwidth}
\textbf{Metode Analisis Data}

\begin{itemize}
\item Analisis Statistik Deskriptif
\item Analisis Crosstabulasi
\item Analisis Biplot
\item[]
\item Perangkat lunak : R untuk Crosstab dan Minitab untuk Biplot
\end{itemize}
\end{columns}
\end{frame}}

\section{Pembahasan}
\begin{frame}[label=current]{Gambaran responden} \vspace{4pt}

\begin{columns}[t,onlytextwidth]
\column{0.5\textwidth}
    \includegraphics[width=.9\linewidth,trim= 1cm .4cm .5cm .8cm,clip]{figures/pieGender}\\

    \includegraphics[width=.9\linewidth,trim= 1cm .4cm .5cm .8cm,clip]{figures/pieUsia}\\
\column{0.5\textwidth}
    \includegraphics[width=.9\linewidth,trim= .8cm .5cm .35cm .6cm,clip]{figures/piekerja}\\

    \includegraphics[width=.9\linewidth,trim= 1cm .3cm .5cm .55cm,clip]{figures/pieSuku}\\
\end{columns}
\end{frame}


%-------------------------------------------------------------------------------------
\begin{comment}
    \input{keynoteco.tex}
\end{comment}
\begin{frame}<beamer>{}
\bibliographystyle{apalike}
{\tiny
\bibliography{biblio.bib}
}
\end{frame}

\end{document}
