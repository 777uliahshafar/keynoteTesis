\documentclass{beamer}

%\includeonlyframes{current}
\input{preamble}

\setbeameroption{hide notes} % Only slides
%\setbeameroption{show only notes} % Only notes
%\setbeameroption{show notes on second screen=right} % Both

\usepackage{beamer_digiPH}
\usepackage{comment}
\usepackage{lipsum}

\title{TESIS}
\subtitle{Preferensi Pengunjung terhadap Ruang di Kawasan Waterfront Senggol Parepare}
\author{Muhammad Uliah Shafar}
\institute{21020119420029\\ \url{777uliahshafar@gmail.com}}

\mode<presentation>{\keywords{Schlüsselwörter durch Komma getrennt}}
\date[28.02.2022]{28 Februari 2022}

\begin{document}

\begin{withoutheadline}
%\begin{frame}
%	% Schriftgröße verkleinern
%	\small
%	% Absatzabstand einstellen
%	\setlength{\parskip}{0.75\baselineskip}
%
%	Dies ist die eLectures-Präsentationsvorlage der Virtuellen PH.
%
%	Die kommenden Folien in \LaTeX\ können und \textbf{\emph{sollen} Sie nach Wunsch adaptieren}, wir bitten Sie aber, sich alle genau anzusehen und durchzulesen! Denn:
%
%	Auf jeder Folie finden Sie Anregungen, Hilfestellungen und Hinweise, sowohl für die inhaltlich-didaktische Vorbereitung als auch die Präsentation online. Zumindest die vorhergehende Titelfolie sollte bitte auf jeden Fall in dieser Form Verwendung finden (Wiedererkennungswert!)
%
%	Bitte senden Sie Ihre überarbeitete PDF-Datei \textbf{vor dem gemeinsamen Testtermin} direkt an Ihre/n Co-Moderator/in. Er/sie bespricht sie mit Ihnen und lädt sie für die eLecture im Raum hoch. Der Kontakt zur Co-Moderation wird aktiv von dieser hergestellt.
%
%	\begin{block}{Zur didaktischen Planung der Stunde:}
%		Eine eLecture-Stunde geht oft sehr schnell vorbei! Gehen Sie daher bitte von max. \SI{75}{\percent} der Zeit für Ihre Präsentation aus, der Rest ist meist durch Einführung und Vorstellung, Interaktion, Rückfragen und Abschluss belegt. Je interaktiver Sie die eLecture halten, desto besser: besprechen Sie sich mit Ihrer CoModeration!
%	\end{block}
%\end{frame}

{\setbeamercolor{background canvas}{bg=digiPH_orange!20}
\begin{frame}
	%\maketitle
	% \maketitle funktioniert auch im Article-Modus
	 \titlepage
\end{frame}
}

\end{withoutheadline}

\miniframesoff
\begin{frame}[label=inhalt]{Topik}
	\tableofcontents
\end{frame}
\miniframeson

\section{Pendahuluan}

{\cca
\begin{frame}[label=tesis]{Latar Belakang}
\onslide<1->{Kawasan waterfront memiliki karakteristik dan perhatian khusus mengingat pentingnya air sebagai sumber kehidupan.}

\onslide<2->{Munculnya tepi laut merupakan permintaan masyarakat terhadap akses ke tepi laut.}

\onslide<3->{Menurut \cite{madureira2018}, tepi laut mampu meningkatkan kualitas hidup masyarakat dengan cara memenuhi kebutuhannya.}

\centering
    \includegraphics[width=0.4\textwidth]{figures/ltr}\\

%	\begin{block}{Zur Didaktik: Ablauf und Lernziele zu Beginn sind sehr wichtig!\\
%		\small Folgendes gilt aber auch für alle anderen Auflistungen:}
%		\end{block}

\notec{}

\end{frame}}

\begin{frame}[label=tesis]{Latar Belakang}

\onslide<1->{Tepi laut Senggol merupakan kawasan pesisir laut di Parepare.}

\onslide<2->{Belakangan ini, terdapat peningkatan dalam sektor kepariwisataan.}

\onslide<3->{Oleh karena itu, tepi laut Senggol sebagai salah satu tempat yang mendapat perhatian untuk pengembangan.}
\end{frame}

{\cca
\begin{frame}[label=current]{Rumusan Masalah} \vspace{4pt}
\onslide<1->{Setelah mengalami penataan, tepi laut Senggol terbagi atas dua macam ruang yang berbeda.}

\onslide<2->{Munculnya perbedaan ruang ini, membuat pengunjung memilih ruang satu daripada lainnya. Sehingga salah satu ruang memungkinkan kurang terpilih yang lambat laun akan terbengkalai.}

\onslide<3->{Agar kedua ruang tersebut memiliki potensi yang sama untuk dipilih, maka mengetahui kebutuhan masyarakat terhadap ruang adalah sangat penting.}

\onslide<4->{Pemahaman terkait preferensi orang terhadap ruang akan menjelaskan kebutuhan masyarakat \citep{madureira2018}}
\end{frame}}

{\cca
\begin{frame}[label=per]{Pertanyaan Penelitian} \vspace{4pt}
\begin{enumerate}
    \item<1-> Ruang apa yang dipilih pengunjung untuk dimanfaatkan sebagai ruang publik?
    \item<2-> Mengapa orang memiliki preferensi pada ruang tersebut?
    \item<3-> Apa saja elemen-elemen yang ada pada ruang publik sehingga pengunjung lebih memilih ruang tersebut?
\end{enumerate}
\notec{
(b) Berdasarkan permasalahan itu, penelitian ini menyelidiki preferensi  masyarakat terhadap ruang di kawasan tepi laut senggol. Maka penelitian ini menjawab sejumlah pertanyaan penelitian sebagai berikut:}
\end{frame}}
%------------------------------------------------------------------------------------

\section{Tinjauan Pustaka}
{\cca
\begin{frame}[label=tin]{Definisi} \vspace{4pt}
\only<1>{\textbf{Tepi laut}

Tepi laut atau waterfront menurut KBBI adalah {\Large wilayah pesisir.} Berdasarkan kamus Amerika Oxford menyebutkan tepi laut adalah "bagian dari kota yang berdampingan dengan sungai, pelabuhan atau danau."\newline }

\only<2>{\textbf{Preferensi}

Menurut Kamus Besar Bahasa Indonesia, preferensi adalah, 1 (hak untuk) didahulukan dan diutamakan daripada yang lain; {\Large prioritas}; 2 pilihan; kecenderungan; kesukaan.\newline}

\only<3>{\textbf{Elemen Ruang}

Elemen ruang diartikan sebagai fisik ruang publik dimana mencakup kualitas, sifat, fitur dari sebuah ruang yang merupakan bagian dari tatanan ruang publik.}
\end{frame}}

\begin{frame}[label=tin]{Instrumen} \vspace{4pt}
\begin{tikzpicture}[node distance=1cm]

    \node (ask) [startstop, text width= 7cm] {\textbf{ASPEK RUANG}\\

\begin{tabular}{ll}
    \tabitem Aksesibilitas & \tabitem Estetika \\
    \tabitem Keamanan & \tabitem Fasilitas\\
\end{tabular}};

    \node (elm) [startstop, below of=ask, yshift= -3cm, text width= 7cm] {\textbf{ELEMEN RUANG}

            \begin{tabular}{ll}
    \tabitem Jumlah pohon & \tabitem Warna bunga \\
    \tabitem Bentuk pohon & \tabitem Jenis kursi\\
    \tabitem Lebar jalan & \tabitem Pencahayaan jalan\\
    \tabitem Permukaan jalan & \\
\end{tabular}};

    \node (jun) [startstop, xshift=4cm, yshift= -1.8cm, text width= 5cm] {Tepi laut Senggol(ruang A dan B)};

\draw [doublearrow] (ask) -- (elm);
\draw [arrow] (ask) -| (jun);
\draw [arrow] (elm) -| (jun);


\end{tikzpicture}

\end{frame}

\section{Metode Penelitian}
{\cca
\begin{frame}[label=met]{Metode Penelitian} \vspace{4pt}
\begin{columns}[t,onlytextwidth]
\fontsize{9pt}{10pt}\selectfont
\column{0.33\textwidth}\onslide<1->{
\textbf{Desain penelitian}
\begin{itemize}
    \item Bersifat kualitatif dan kuantitatif
    \item Metode penelitian : Survei
\end{itemize}

\textbf{Metode Pengumpulan data}
\begin{itemize}
    \item Metode survei: Survei kualitatif dan kuantitatif
    \item Objek penelitian : Pengunjung tepi laut Senggol
\end{itemize}

}
\column{0.33\textwidth}\onslide<2->{
\textbf{Metode pengambilan sampel}
\begin{itemize}
    \item Teknik penyampelan adalah probabilitas dengan metode \textit{stratified random sampling}.
    \item Kriteria
    \item Jumlah sampel : 85 observasi
\end{itemize}
}
\column{0.33\textwidth}\onslide<3->{
\textbf{Metode Analisis Data}

\begin{itemize}
\item Analisis Statistik Deskriptif
\item Analisis Crosstabulasi
\item Analisis Biplot
\item[] \item Perangkat lunak : R untuk Crosstab dan Minitab untuk Biplot
\end{itemize}}
\end{columns}
\end{frame}}

\section{Pembahasan}
\begin{frame}[label=current]{Gambaran responden} \vspace{4pt}

\begin{columns}[t,onlytextwidth]
\column{0.5\textwidth}
    \includegraphics[width=.9\linewidth,trim= 1cm .4cm .5cm .8cm,clip]{figures/pieGender}\\

   \includegraphics[width=.9\linewidth,trim= 1cm .4cm .5cm .8cm,clip]{figures/pieUsia}\\
\column{0.5\textwidth}
\includegraphics[width=.9\linewidth,trim= .8cm .5cm .35cm .6cm,clip]{figures/piekerja}\\

\includegraphics[width=.9\linewidth,trim= 1cm .3cm .5cm .55cm,clip]{figures/pieSuku}\\
\end{columns}
\end{frame}

\begin{frame}[t,label=current]{Ruang yang paling disukai} \vspace{4pt}
\begin{center}
    \includegraphics[width=.65\textwidth,trim= 1cm .3cm .5cm .6cm,clip]{figures/pieRuang}\\
\end{center}
\end{frame}

\begin{frame}[t,label=current]{Keragaman preferensi terhadap ruang}
\centering
\begin{minipage}{.55\textwidth}
\small
\begin{tabularx}{\linewidth}{ldd}
  \toprule
\multirow{2}{*}{Ruang} & \multicolumn{2}{c}{Gender} \\
 \cline{2-3} & \multicolumn{1}{c}{laki-laki } & \multicolumn{1}{c}{perempuan } \\
  \midrule
 ruang a  & 36 & 20 \\
  col \% & 62.1 & 74.1 \\
  \hline
 ruang b  & 22 &  7 \\
  col \% & 37.9 & 25.9 \\
   \hline
 \bottomrule
\end{tabularx}

\end{minipage}
\begin{minipage}{.55\textwidth}

\small
\begin{tabularx}{1\linewidth}{lddd}
  \toprule
\multirow{2}{*}{Ruang} & \multicolumn{3}{c}{Kelompok Usia} \\
 \cline{2-4} & \multicolumn{1}{c}{remaja} & \multicolumn{1}{c}{dewasa} & \multicolumn{1}{c}{manula} \\
  \midrule
 ruang a  & 25 & 20 & 11 \\
  col \% & 83.3 & 57.1 & 55.0 \\
  \hline
 ruang b  &  5 & 15 &  9 \\
  col \% & 16.7 & 42.9 & 45.0 \\
   \hline
 \bottomrule
\end{tabularx}


\end{minipage}

\end{frame}

\begin{frame}[t,label=current]{Keragaman preferensi terhadap ruang}
\centering
\begin{minipage}{.55\textwidth}
\small
\begin{tabularx}{\linewidth}{lddd}
  \toprule
\multirow{2}{*}{Ruang} & \multicolumn{3}{c}{Kelompok Usia} \\
 \cline{2-4} & \multicolumn{1}{c}{remaja} & \multicolumn{1}{c}{dewasa} & \multicolumn{1}{c}{manula} \\
  \midrule
 ruang a  & 25 & 20 & 11 \\
  col \% & 83.3 & 57.1 & 55.0 \\
  \hline
 ruang b  &  5 & 15 &  9 \\
  col \% & 16.7 & 42.9 & 45.0 \\
   \hline
 \bottomrule
\end{tabularx}

\end{minipage}
\begin{minipage}{.99\textwidth}

\footnotesize
\begin{tabularx}{1\linewidth}{lddddd}
  \toprule
\multirow{2}{*}{Ruang} & \multicolumn{5}{c}{Pekerjaan} \\
 \cline{2-6} & \multicolumn{1}{c}{ belum bekerja } & \multicolumn{1}{c}{ karyawan      } & \multicolumn{1}{c}{ lainnya       } & \multicolumn{1}{c}{ pelajar       } & \multicolumn{1}{c}{ wiraswasta    } \\
  \midrule
 ruang a  &  6 & 17 &  7 & 16 & 10 \\
  col \% & 100.0 & 58.6 & 58.3 & 76.2 & 58.8 \\
  \hline
 ruang b  &  0 & 12 &  5 &  5 &  7 \\
  col \% & 0.0 & 41.4 & 41.7 & 23.8 & 41.2 \\
   \hline
 \bottomrule
\end{tabularx}


\end{minipage}

\end{frame}

\begin{frame}[label=current]{Preferensi terhadap aspek} \vspace{4pt}
    \small
\begin{tabularx}{\textwidth}{r p{4.2em}
    p{4.2em}p{4.2em}p{4.2em}p{5em}}
    \hline
\bfseries\diagbox[innerleftsep=10pt,innerrightsep=3pt,width=7em, height=2.2cm]{Ruang}{Aspek\\Ruang}&
 {\rotatebox[origin=c]{90}{\parbox[c]{2.2cm}{\textbf{Aksesibili-\\tas}}}} & {\rotatebox[origin=c]{90}{\parbox[c]{2.2cm}{\textbf{Keamanan}}}} & {\rotatebox[origin=c]{90}{\parbox[c]{2.2cm}{\textbf{Estetika}}}} & {\rotatebox[origin=c]{90}{\parbox[c]{2.2cm}{\textbf{Fasilitas}}}} &{\rotatebox[origin=c]{90}{\parbox[c]{2.2cm}{\textbf{Total}}}} \\
 \toprule
Ruang A  & 32 (21\%) & 5 (3\%)   & 37 (25\%) & 30 (20\%) & 104 (69\%) \\
Ruang B  & 15 (10\%) & 2 (2\%)   & 9 (6\%) & 20 (13\%) & 46 (31\%) \\
Total  & 47 (31\%) & 7 (5\%)  & 46 (31\%) & 50 (33\%) &150 (100\%) \\

 \bottomrule
\end{tabularx}
\end{frame}

\begin{frame}[label=current]{Deskripsi responden}
\onslide<1->{\begin{quoting} Ruang ini memberi kesan lapang tapi tetap memiliki estetika sehingga membuat kita menjadi nyaman ketika berada di ruang tersebut.\end{quoting}}
\onslide<2->{\begin{quoting} Ruang ini terlihat lebih rapi, pedestrian yang lumayan luas dan tidak banyak kendaraan yang berlalu lalang. \end{quoting}}
\onslide<3->{\begin{quoting} Karena bersih, rapi dan memiliki pemandangan yang indah. \end{quoting}}
\end{frame}

\begin{frame}[label=current]{Keragaman preferensi terhadap aspek}
\begin{figure}[htpb]
    \centering
    \includegraphics[width=.65\textwidth,trim={.4cm .3cm .4cm .1cm},clip]{figures/bra.png}
    \caption{Keragaman Preferensi terhadap Aspek dari setiap ruang}
    \caption*{Sumber: Analisis, 2022}
\end{figure}
\end{frame}

\begin{frame}[label=current]
\footnotesize
\begin{tabular}[h!]{rP{5.2em}P{5.2em}P{5.4em} }
\hline
\bfseries\diagbox[width=8em, height=1.5cm]{Elemen \\ruang}{Kelompok}&{\bfseries\parbox[c][1.5cm]{\textwidth}{Ruang a}} &\textbf{Ruang b} & \textbf{Total}\\
\toprule

Sedikit pohon  & 6 (7\%)  & 3 (4\%)    &9(4\%) \\
Beberapa pohon  &10 (12\%)  & 1 (1\%)    &11 (5\%) \\

Cukup rindang  &2 (2\%)  & 28 (33\%)    &30(14\%) \\
Sangat rindang  &4 (5\%)  & 6 (7\%)    &10(4) \\

1-3m  &3 (4\%)  & 6 (7\%)   &9(4\%) \\
> 3m  &35 (41\%)  & 7 (8\%)    &42(19\%)\\

Paving  &7 (8\%)  & 4 (5\%)    &11(5\%) \\
Aspal  &12(14\%)  & 1(1\%)    &13(6\%) \\
Keramik  &12(14\%)  & 1(1\%)    &13(6\%) \\

1 atau 2 warna  &2(2\%)  & 0    &2(1\%) \\
lebih 3 warna  &5(6\%)  & 1(1\%)    &6(3\%) \\

kursi bergerak  &6(7\%)  & 0    &6(3\%)\\
kursi dinding  &3(4\%)  & 2(2\%)    &5(2\%) \\

pencahayaan cukup   &5 (6\%)  & 31 (36\%)    &36(16\%) \\
pencahayaan tinggi   &11(13\%)  & 6 (7\%)    &17(8\%)\\

Total  & 123 (56\%)  & 97 (44\%) & 220 (100\%)   \\
 \bottomrule
\end{tabular}


\end{frame}

\begin{frame}[label=current]{Preferensi terhadap kelompok elemen}
\begin{tabular}[ht]{rP{5em}P{5em}P{5.5em} }
\hline
\bfseries\diagbox[innerleftsep=8pt,innerrightsep=3pt,width=10em, height=2.5cm]{Elemen \\ruang}{Kelompok}&{\bfseries\parbox[c][2.5cm]{\textwidth}{Ruang a}} &\textbf{Ruang b} & \textbf{Total}\\


\toprule

Jumlah pohon  & 16 (7\%)  & 4 (2\%) & 20 (9\%)    \\
Bentuk pohon  & 30 (14\%)  & 10 (4\%) & 40 (18\%)   \\
Lebar jalan  & 38 (17\%)  & 13 (6\%) &51 (23\%)   \\
Permukaan jalan  & 31 (14\%)  & 6 (3\%) & 37 (17\%)   \\
Warna bunga & 7 (3\%)  & 1   & 8 (4\%)  \\
Jenis kursi  & 9 (4\%)  & 2 (1\%)  & 11 (5\%)  \\
Tingkat cahaya  & 36 (16\%)  & 17 (8\%) & 53 (24\%)   \\

Total  & 167 (76\%)  & 53 (24\%) & 220 (100\%)   \\

\bottomrule

\end{tabular}


\end{frame}

\begin{frame}[label=current]{Keragaman preferensi terhadap klp elemen}
\begin{figure}[htpb]
    \centering
    \includegraphics[width=.65\textwidth,trim={.5cm .3cm .5cm .1cm},clip]{figures/bre.png}
    \caption{Keragaman Preferensi terhadap elemen dari setiap ruang}
    \caption*{Sumber: Analisis, 2022}
    \label{fig:bre}
\end{figure}
\end{frame}

\begin{frame}[label=current]{Kesimpulan} \vspace{4pt}
Setelah melakukan analisis, sejumlah pertanyaan penelitian terjawab yang kemudian disimpulkan sebagai berikut:

\begin{itemize}
    \item<1-> Hasil dari studi ini mengindikasikan ruang A paling dipreferensi di tepi laut Senggol.
    \item<2->Aspek fasilitas adalah aspek yang paling disukai orang dalam memilih ruang. Oleh karena itu, aspek fasilitas seharusnya diutamakan dalam desain dan pengelolaan ruang terbuka tepi laut.
    \item<3-> Elemen yang paling disukai oleh pengunjung adalah lebar jalan. Ini mendukung aspek aksesibilitas sebagai urutan kedua alasan memilih ruang.
\end{itemize}


\end{frame}

%-------------------------------------------------------------------------------------
\begin{comment}
    \input{keynoteco.tex}
\end{comment}
\begin{frame}<beamer>{}
\bibliographystyle{apalike}
{\tiny
\bibliography{biblio.bib}
}
\end{frame}

\end{document}
