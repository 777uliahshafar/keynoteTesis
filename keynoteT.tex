\documentclass{beamer}

%\includeonlyframes{obj}
\input{preamble}

%\setbeameroption{hide notes} % Only slides
%\setbeameroption{show only notes} % Only notes
\setbeameroption{show notes on second screen=right} % Both

\usepackage{beamer_digiPH}
\usepackage{comment}
\usepackage{lipsum}

\title{PRA TESIS}
\subtitle{Preferensi Pengunjung terhadap Ruang di Kawasan Waterfront Senggol Parepare}
\author{Muhammad Uliah Shafar}
\institute{21020119420029\\ \url{777uliahshafar@gmail.com}}

\mode<presentation>{\keywords{Schlüsselwörter durch Komma getrennt}}
\date[09.04.2018]{9 April 2018}

\begin{document}

\begin{withoutheadline}
%\begin{frame}
%	% Schriftgröße verkleinern
%	\small
%	% Absatzabstand einstellen
%	\setlength{\parskip}{0.75\baselineskip}
%
%	Dies ist die eLectures-Präsentationsvorlage der Virtuellen PH.
%
%	Die kommenden Folien in \LaTeX\ können und \textbf{\emph{sollen} Sie nach Wunsch adaptieren}, wir bitten Sie aber, sich alle genau anzusehen und durchzulesen! Denn:
%
%	Auf jeder Folie finden Sie Anregungen, Hilfestellungen und Hinweise, sowohl für die inhaltlich-didaktische Vorbereitung als auch die Präsentation online. Zumindest die vorhergehende Titelfolie sollte bitte auf jeden Fall in dieser Form Verwendung finden (Wiedererkennungswert!)
%
%	Bitte senden Sie Ihre überarbeitete PDF-Datei \textbf{vor dem gemeinsamen Testtermin} direkt an Ihre/n Co-Moderator/in. Er/sie bespricht sie mit Ihnen und lädt sie für die eLecture im Raum hoch. Der Kontakt zur Co-Moderation wird aktiv von dieser hergestellt.
%
%	\begin{block}{Zur didaktischen Planung der Stunde:}
%		Eine eLecture-Stunde geht oft sehr schnell vorbei! Gehen Sie daher bitte von max. \SI{75}{\percent} der Zeit für Ihre Präsentation aus, der Rest ist meist durch Einführung und Vorstellung, Interaktion, Rückfragen und Abschluss belegt. Je interaktiver Sie die eLecture halten, desto besser: besprechen Sie sich mit Ihrer CoModeration!
%	\end{block}
%\end{frame}

{\setbeamercolor{background canvas}{bg=digiPH_orange!20}
\begin{frame}
	%\maketitle
	% \maketitle funktioniert auch im Article-Modus
	 \titlepage
\end{frame}
}

\end{withoutheadline}

\miniframesoff
\begin{frame}[label=inhalt]{Overview}
	\tableofcontents
\end{frame}
\miniframeson

\section{Pendahuluan}

{\ca
\begin{frame}[label=intro]{Latar Belakang}
Kawasan waterfront memiliki karakteristik dan perhatian khusus mengingat pentingnya air sebagai sumber kehidupan.

    \includegraphics[width=0.4\textwidth]{figures/ltr}\\
        {\tiny \textcolor{digiPH_darkblue}{Sumber: Penulis, \href{https://creativecommons.org/licenses/by/3.0/at/}{CC BY}}}

%	\begin{block}{Zur Didaktik: Ablauf und Lernziele zu Beginn sind sehr wichtig!\\
%		\small Folgendes gilt aber auch für alle anderen Auflistungen:}
%		\end{block}

\notec{Untuk mencapai tujuan tersebut, pengembangan tepi laut adalah sangat penting. Menurut hussein2014, pengembangan tepi laut yang baik adalah yang mempertimbangkan keberagaman, interaksi komunitas, kenyamanan dan keamanan, lingkungan dan keberlanjutan.}

\notec{Menurut breen1994, tekanan pada ruang kota dan infrastruktur, kebutuhan atas kualitas lingkungan, dan ketersediaan ruang tepi laut yang terbengkalai menjadi alasan pengembangan ulang kawasan tepi laut sebagai solusi yang pas.
}
\end{frame}}

\begin{frame}[label=intro]
Pengembangan tepi laut berkelanjutan antara lain:
\begin{itemize}
    \item Proyek reklamasi di Makassar dan Manado
    \item Pengembangan tepi laut tahun 1995 sepanjang 32km
    \item Desain lanskap \textit{riverside} Cikapundung
\end{itemize}

\end{frame}

\begin{frame}[label=intro]{Kota Parepare}
Kota Parepare merupakan kota yang terletak di Provinsi Sulawesi Selatan. Peningkatan jumlah penduduk di Parepare berkisar 2\%, pada tahun 2019 Parepare memiliki penduduk sebanyak 145.178 orang.

\notec{Saat ini, Kota Parepare sedang melakukan sejumlah kemajuan di bidang pariwisata. Salah satunya adalah revitalisasi tepi laut senggol.}

\notec{Pengembangan tepi laut ini bertujuan agar mampu mendorong jumlah pengunjung pada tempat wisata tersebut, sebagaimana  menjelaskan keberhasilan suatu tepi laut ditandai dengan pengembangannya membawa masyarakat dan pengunjung untuk datang ke pesisir.}

\end{frame}

\begin{frame}[label=intro]{Kota Parepare} \vspace{4pt}
 	\begin{center}
		\includegraphics[width=\textwidth]{figures/pr}
		{\tiny \textcolor{digiPH_darkorange}{Sumber: Penulis, \href{https://creativecommons.org/licenses/by/3.0/at/}{CC BY}}}
	\end{center}

\notec{
Kawasan tepi laut senggol terbentang dari Pelabuhan Nusantara hingga Pasar Senggol sepanjang sekitar 300 meter. Sepanjang garis pantai tersebut terbentuk sejumlah ruang dengan karakteristik yang berbeda. Pengembangan yang terjadi di kawasan tersebut untuk merespon konsep kota Parepare sebagai kota Pariwisata.}

\end{frame}

{\ca
\begin{frame}[label=intro]
Ruang menjadi tempat yang dapat mengakomodasi masyarakat untuk meningkatkan kualitas hidup mereka, dengan cara memenuhi kebutuhannya.
Mencari tahu preferensi ruang dari masyarakat dapat membantu menyediakan dan mengelola pengembangan untuk memenuhi kebutuhan masyarakat secara efektif.

\centering
    \includegraphics[width=0.35\textwidth]{figures/pref}\\
        {\tiny \textcolor{digiPH_darkblue}{Sumber: Penulis, \href{https://creativecommons.org/licenses/by/3.0/at/}{CC BY}}}

\notec{preferensi adalah kecenderungan untuk memilih sesuatu yang lebih disukai daripada yang lain. Sejumlah atribut pada ruang tersebut menjadi alasan dalam pemilihan ruang di kawasan tepi laut.
}
\end{frame}}

\begin{frame}[label=intro]{State of the Art}
Penelitian terkait preferensi telah banyak dibahas, kebanyakan membahas tentang preferensi di ruang publik tengah kota seperti taman kota\newline

{\Large Namun terlepas dari studi berkaitan dengan preferensi ruang publik, sepengetahuan penulis hanya sedikit yang membahas tentang preferensi pengunjung terhadap ruang khususnya di kawasan tepi laut.}

\notec{Penelitian terkait preferensi antara lain:
\begin{itemize}
    \item Preferensi terhadap penataan permukiman nelayan kumuh
    \item Preferensi pengguna terhadap kualitas taman kota sebagai ruang publik
    \item Preferensi masyarakat terhadap taman kota di pusat kota tangerang
\end{itemize}}

\end{frame}


\begin{frame}[label=per]{Permasalahan} \vspace{4pt}

Ruang yang menunjukkan faktor perseptual yang lengkap tampaknya kurang dicenderungi oleh pengunjung. Berbeda dengan ruang yang atributnya kurang menunjukkan tingkat kecenderungan yang besar. Ruang yang memiliki atribut lingkungan yang lengkap disebut ruang A sedangkan ruang satunya yang memiliki atribut yang kurang disebut ruang B.

\notec{Pada tahun 2011, kota Parepare memulai perencanaan penataan kawasan tepi laut senggol. Penataan ini memunculkan dua ruang yang memiliki lingkungan dengan atribut yang berbeda. Ruang publik yang berhasil ditandai dengan kehadiran orang (Carr et al., 1992; Hoyle, 2001). Menurut Swanwick (2009), atribut lingkungan (se­ perti keberagaman, kontras dan warna) mendasari presepsi dan kesukaan orang terhadap lanskap tertentu berkaitan dengan lanskap secara keseluruhan. Akan tetapi yang terjadi di tepi laut Senggol,}

\end{frame}

{\ca
\begin{frame}[label=per]{Pertanyaan Penelitian} \vspace{4pt}
\begin{enumerate}
    \item Apa {\Large preferensi ruang masyarakat} di kawasan tepi laut Senggol?
    \item Apa {\Large atribut yang paling penting} terhadap pemilihan ruang? Apakah kepentingannya bervariasi diantara ruang­ruang?
    % Apakah fitur ruang tepi laut mempunyai hubungan terhadap pemilihan ruang?
\end{enumerate}
\notec{
Berdasarkan permasalahan itu, penelitian ini menyelidiki preferensi  masyarakat terhadap ruang di kawasan tepi laut senggol. Maka penelitian ini menjawab sejumlah pertanyaan penelitian sebagai berikut:}
\end{frame}}


\begin{frame}[label=per]{Tujuan Penelitian} \vspace{4pt}
\begin{enumerate}
    \item Untuk mengetahui preferensi pengunjung terhadap ruang berdasarkan atribut ruang publik tepi laut.
    \item Untuk menyelidiki atribut terkait preferensi masyarakat terhadap ruang di tepi laut Senggol.
\end{enumerate}
\end{frame}



\begin{frame}[label=per]{Manfaat Penelitian} \vspace{4pt}
\begin{enumerate}
\item Memberikan masukan desain secara keseluruhan berdasarkan preferensi ruang masyarakat.
\item Mendukung penelitian selanjutanya dalam ranah preferensi ruang tepi laut.
\item Memberikan panduan terhadap pengembangan tepi laut dimanapun dalam melibatkan masyarakat menggunakan informasi preferensinya.
\end{enumerate}
\end{frame}



%------------------------------------------------------------------------------------

\section{Tinjauan Pustaka}
{\ca
\begin{frame}[label=tin]{Definisi} \vspace{4pt}
Tepi laut atau waterfront menurut KBBI adalah {\Large wilayah pesisir.} Berdasarkan kamus Amerika Oxford menyebutkan tepi laut adalah "bagian dari kota yang berdampingan dengan sungai, pelabuhan atau danau."\newline

Tepi laut adalah kawasan yang dinamis {\Large suatu kota} tempat bertemunya daratan dan perairan\newline

Menurut Kamus Besar Bahasa Indonesia, preferensi adalah, 1 (hak untuk) didahulukan dan diutamakan daripada yang lain; {\Large prioritas}; 2 pilihan; kecenderungan; kesukaan.


\end{frame}}

\begin{frame}[label=tin]{Definisi} \vspace{4pt}
Atribut ruang, istilah lain elemen lingkungan adalah fitur atau kualitas dari lingkungan yang mana bagian dari tatanan \textit{(settings)} ruang publik.


\Large Karakter pengguna adalah sesuatu yang terkait dengan keadaan \textit{socio-demografi} seperti {\huge umur, ras, gender dan status pekerjaan.}
\notec{itu adalah bagian dari \textit{setting} {\Large ruang publik }(seperti \textit{waterfront \& taman}) yang khas dan konteks lingkungan yang lebih luas.
}

\end{frame}

\begin{frame}[label=tin]{Instrumen} \vspace{4pt}
\begin{columns}[onlytextwidth]

\column{0.5\textwidth}
\begin{enumerate}
  \setlength\itemsep{.2em}
    \item Aksesibilitas
    \item Estetika
    \item Keamanan dan kesalamatan
    \item Fasilitas

\end{enumerate}
\column{0.5\textwidth}
\begin{enumerate}
  \setlength\itemsep{.2em}
    \item Umur
    \item Gender
    \item Ras
    \item Status Pekerjaan
\end{enumerate}

\end{columns}

\begin{tikzpicture}[node distance=2cm]
	\node (fit) [startstop, text width= 5cm] {\textbf{Atribut Ruang}\\\small(aksesibilitas, estetika, keamanan, dan fasilitas)};

	\node (soc) [startstop, below of=fit, text width= 4cm, yshift=-.5cm, ] {\textbf{Karakter Pengunjung}\\\small(umur,gender,ras dan status kegiatan ekonomi)};

	\node (pre) [startstop, right of=fit, text width= 4cm,yshift=-.5cm, xshift=4cm] {\textbf{Preferensi Pengunjung terhadap Ruang}\\ Ruang A atau Ruang B};

\draw [arrow] (fit) -- (pre);
\draw [arrow] (soc) -- (pre);

\end{tikzpicture}

\end{frame}

\begin{frame}[label=current]

\begin{description}
    \item[\bf Aksesibilitas]: Kemudahan dikenali, Pencapaian melalui jalan, Bebas hambatan, Penggunaan yang nyaman, Pengawasan, Ketersediaan parkir.
    \item[\bf Keamanan dan keselamatan]: Derajat ketertutupan, Keramaian, Gangguan fisik dan sosial,Privatisasi, Pencahayaan.
    \item[\bf Estetika]: Derajat ketertutupan, Susunan \textit{(order)}, Pola material, Kualitas dan tatanan furnitur jalan, Pemeliharaan.
    \item[\bf Fasilitas]: Memberi rasa nyaman, Memberi relaksasi, Mendukung interaksi sosial ,Mendukung privasi, Keragaman penggunaan.

\end{description}
\end{frame}

\begin{frame}[label=current]\vspace{4pt}
\begin{description}
    \item[\bf Umur]: 18-44, 45-64, >65
    \item[Gender]: Laki-laki, Perempuan
    \item[Ras]: Bugis, Bukan bugis
    \item[Status pekerjaan]: Karyawan, Wiraswasta, Pengangguran, Pelajar, Pensiun
\end{description}
\end{frame}

\section{Metode Penelitian}
{\ca
\begin{frame}[label=met]{Metode Penelitian} \vspace{4pt}
Penelitian ini menggunakan metode kualitatif deskriptif...
\newline
Lokasi dan Waktu Penelitian:
\begin{enumerate}
    \item Hari Senin-Jumat pada pagi hari pukul 06.00 - 09.00
    \item Hari Sabtu dan Minggu pukul 06.00-11.00
\end{enumerate}

\notec{Penelitian ini menggunakan metode kualitatif eksploratif untuk mendeskripsikan atau menggambarkan preferensi pengunjung terhadap ruang dan atribut ruang.}
\notec{Metode lain yang digunakan adalah metode statistik sederhana dengan menggunakan crosstab untuk mengetahui hubungan atribut dan preferensi masyarakat terhadap ruang.}
\notec{penelitian ini juga menggunakan metode analisis distribusi untuk memahami dominasi frekuensi kepentingan sebuah atribut ruang.}

\end{frame}}

\begin{frame}[label=met]{Data} \vspace{4pt}
\textbf{Penentuan Sampel}
Berdasarkan teori pengambilan sampel dan populasi, jumlah sampel yang diambil pada penelitian ini adalah {\Large 99 orang.}

\textbf{Pengumpulan Data Primer}
\begin{itemize}
    \item Pengamatan langsung
    \item Wawancara
    \item Kuesioner
\end{itemize}
\notec{Peneliti menggunakan sampel jenis non­random/purposive }
\notec{Jenis sampel ini mengidentifikasi dan memilih individu atau kelompok yang mampu dan mengetahui fenomena yang sedang diteliti. Pemilihan karakter utama subjek adalah orang yang berusia 18 tahun keatas dan pengunjung tepi laut senggol.}
\notec{Tujuan observasi adalah menggambarkan tempat, aktivitas, pelaku, dan makna yang terjadi dalam proses observasi}
\notec{Penelitian ini menggunakan jenis kuesioner langsung tertutup}
\end{frame}

\begin{frame}[label=met]{Analisis Data} \vspace{4pt}
Mengidentifikasi dan membaca hasil pengelolaan data untuk menganalisis dan membahas fenomena yang diteliti.

\begin{enumerate}
    \item Reduksi data
    \item Penyajian data
    \item Penarikan kesimpulan
\end{enumerate}

Analisis data statistik penelitian ini menggunakan \textit{crosstab}
\notec{analisis data yang menggunakan statistik \textit{(crosstab)}, didapatkan hubungan antara atribut dan orang dengan ruang yang terpilih.}
\end{frame}

%-------------------------------------------------------------------------------------

\section{Objek Penelitian}
{\ca
\begin{frame}[label=obj]{Objek Penelitian} \vspace{4pt}
{\Large Kota Parepare} adalah kota tempat kelahiran presiden ke-3 Indonesia BJ Habibie. Istilah tersebut menjadi ikon bagi Parepare untuk {\Large memajukan aspek kepariwisataan.} Peningkatan kepariwisataan Parepare mendorong kemajuan kawasan pesisir.
\notec{
    Sebelah Utara : Kabupaten Pinrang,
    Sebelah Timur : Kabupaten Sidenreng Rappang,
    Sebelah Barat : Selat Makassar atau Teluk Parepare,
    Sebelah Selatan : Kabupaten Barru}
\end{frame}}

\begin{frame}[label=obj]{Lokasi Penelitian} \vspace{4pt}
\begin{center}
    \includegraphics[width=0.8\textwidth]{figures/lokzi}\\
        {\tiny \textcolor{digiPH_darkblue}{Sumber: Penulis, \href{https://creativecommons.org/licenses/by/3.0/at/}{CC BY}}}
\end{center}

\notec{Berdasarkan RPI2JM 2017-2021 kota Parepare, kawasan yang biasa disebut senggol ini merupakan kawasan strategis kota (KSK) untuk pengembangan PKL dalam kepentingan pertumbuhan ekonomi.
}
\end{frame}

\section{Atribut Ruang}
{\ca
\begin{frame}[label=obj]{Akses} \vspace{4pt}
Ruang A\\
\begin{columns}[onlytextwidth]
\column{0.5\textwidth}
    \includegraphics[width=.95\textwidth]{figures/akses1}
\column{0.5\textwidth}
    \includegraphics[width=.95\textwidth]{figures/akses1(2)}
\end{columns}

\notec{Nilai akses ditandai dengan lebar jalan dan kedekatan fasilitas pada suatu ruang.}
\notec{Ruang A memiliki lebar jalan yang cukup besar. Akses ini terbilang cukup memadai dengan penambahan penerangan dan paving.}


\end{frame}}

\begin{frame}[label=obj]{Akses} \vspace{4pt}
Ruang B\\
    \includegraphics[width=.6\textwidth]{figures/akses2}

\notec{Penerapan akses pada ruang B masih terbilang kurang. Seperti pada masalah umum ruang publik, tempat ini tidak menyediakan pedestrian khusus.}
\end{frame}

\begin{frame}[label=obj]{Estetika} \vspace{4pt}
Ruang A\\
\begin{columns}[onlytextwidth]
\column{0.5\textwidth}
    \includegraphics[width=.95\textwidth]{figures/est1}
\column{0.5\textwidth}
    \includegraphics[width=.95\textwidth]{figures/est1(2)}
\end{columns}
\notec{atribut estetika dominan didukung oleh elemen alami yaitu kerapatan vegetasi dan kehadiran tumbuhan.}
\notec{Letak pohon tersebut lebih banyak di beton pembatas, dimana kehadiran tumbuh-tumbuhan pada tempat ini terbilang sedikit. Terdapat sekitar 20 pohon yang ada pada ruang A ini.}
\end{frame}

\begin{frame}[label=obj]{Estetika} \vspace{4pt}
Ruang B\\
    \includegraphics[width=.58\textwidth]{figures/est2}
\notec{Estetika ruang B menunjukkan kerapatan vegetasi yang lebih dari ruang A. Jarak antar pohon ada yang hanya sekitar 3 meter dan ada juga yang 5 meter.
}
\end{frame}


\begin{frame}[label=obj]{Estetika} \vspace{4pt}
Ruang A\\
\begin{columns}[onlytextwidth]
\column{0.5\textwidth}
    \includegraphics[width=.95\textwidth]{figures/pem1}
\column{0.5\textwidth}
    \includegraphics[width=.95\textwidth]{figures/pem1(4)}
\end{columns}

\notec{Pemeliharaan pada ruang publik lebih memperhatikan kondisi berlanjut \textit{(continue)} yang ada di lapangan seperti kondisi rumput, dinding(grafiti), dan sampah.}

\end{frame}

\begin{frame}[label=obj]{Estetika} \vspace{4pt}
Ruang B\\
    \includegraphics[width=.9\textwidth]{figures/pem2}
\end{frame}


\begin{frame}[label=obj]{Keamanan} \vspace{4pt}
Ruang A\\
    \includegraphics[width=.9\textwidth]{figures/amn1}
\notec{Peneliti meninjau ketertutupan dan tingkat keramaian sebagai ukuran keamanan. Ketertutupan sebuah ruang dapat dipengaruhi oleh karakteristik ruang dan fungsi bangunan, ketinggian dan \textit{setback} vegetasi, pagar dan bangunan.}

\notec{Ketertutupan pada ruang A didominasi oleh kios penjual dengan tinggi bangunan 5 meter dan lebar pedestrian sekitar 4 meter. Sehingga tempat ini hampir memenuhi proporsi 1:1 atau cukup terasa ketertutupan berdasarkan teori GLC. Selain derajat ketertutupan, keramaian juga menjadi fokus dari keamanan. Keramaian pada ruang A pada hari-hari biasa ataupun hari libur terlihat seperti gambar}
\end{frame}

\begin{frame}[label=obj]{Fasilitas} \vspace{4pt}
Ruang A\\
\begin{columns}[onlytextwidth]
\column{0.5\textwidth}
    \includegraphics[width=.95\textwidth]{figures/fb1(2)}
\column{0.5\textwidth}
    \includegraphics[width=.95\textwidth]{figures/fb1(3)}
\end{columns}
\notec{Fasilitas pada setiap ruang publik penulis kategorikan sebagai fasilitas-fasilitas buatan, hijau, dan biru}

\notec{Berbeda dengan ruang A, ruang B memiliki fasilitas yang kurang beragam. Ada sebuah patung adipura yang terletak di ujung pesisir ini. Padahal patung adipura tersebut berperan juga sebagai ruang terbuka dimana sangat dibutuhkan masyarakat.}

\end{frame}

\begin{frame}[label=obj]{Fasilitas} \vspace{4pt}
Ruang B\\
\begin{columns}[onlytextwidth]
\column{0.5\textwidth}
    \includegraphics[width=.95\textwidth]{figures/fb2(4)}
\column{0.5\textwidth}
    \includegraphics[width=.95\textwidth]{figures/fb2(5)}
\end{columns}
\end{frame}






\begin{comment}
    \input{keynoteco.tex}
\end{comment}



\begin{frame}<beamer>{}
\bibliographystyle{apalike}
{\tiny
\bibliography{biblio.bib}
}
\end{frame}

\end{document}
