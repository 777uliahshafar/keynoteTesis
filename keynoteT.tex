\documentclass{beamer}

\input{preamble}

\setbeameroption{hide notes} % Only slides
%\setbeameroption{show only notes} % Only notes
%\setbeameroption{show notes on second screen=right} % Both

\usepackage{beamer_digiPH}
\usepackage{comment}
\usepackage{lipsum}

%\title{\textbf{ Universitas Diponegoro}}
%\subtitle{\textbf{Preferensi Ruang Kawasan \textit{Waterfront} Senggol Parepare}}
%\author{Muhammad Uliah Shafar \texorpdfstring{\\ \\} \small Pembimbing: \texorpdfstring{\\} DDr. Ars. Ir. Wijayanti, M.Eng \texorpdfstring{\\} PProf.Dr. Ir. Atik Suprapti, M.T.}

\title{PRA TESIS}
\subtitle{Preferensi Ruang Kawasan Tepi Laut Senggol Parepare}
\author{Muhammad Uliah Shafar}
%\date{\today}
\institute{21020119420029\\ \url{777uliahshafar@gmail.com}}

%\institute[Lehrstuhl]{
%	Name des Lehrstuhls \\
%	Name des Instituts/Fachbereichs \\
%}
\mode<presentation>{\keywords{Schlüsselwörter durch Komma getrennt}}
\date[09.04.2018]{9 April 2018}

\begin{document}

\begin{withoutheadline}
%\begin{frame}
%	% Schriftgröße verkleinern
%	\small
%	% Absatzabstand einstellen
%	\setlength{\parskip}{0.75\baselineskip}
%
%	Dies ist die eLectures-Präsentationsvorlage der Virtuellen PH.
%
%	Die kommenden Folien in \LaTeX\ können und \textbf{\emph{sollen} Sie nach Wunsch adaptieren}, wir bitten Sie aber, sich alle genau anzusehen und durchzulesen! Denn:
%
%	Auf jeder Folie finden Sie Anregungen, Hilfestellungen und Hinweise, sowohl für die inhaltlich-didaktische Vorbereitung als auch die Präsentation online. Zumindest die vorhergehende Titelfolie sollte bitte auf jeden Fall in dieser Form Verwendung finden (Wiedererkennungswert!)
%
%	Bitte senden Sie Ihre überarbeitete PDF-Datei \textbf{vor dem gemeinsamen Testtermin} direkt an Ihre/n Co-Moderator/in. Er/sie bespricht sie mit Ihnen und lädt sie für die eLecture im Raum hoch. Der Kontakt zur Co-Moderation wird aktiv von dieser hergestellt.
%
%	\begin{block}{Zur didaktischen Planung der Stunde:}
%		Eine eLecture-Stunde geht oft sehr schnell vorbei! Gehen Sie daher bitte von max. \SI{75}{\percent} der Zeit für Ihre Präsentation aus, der Rest ist meist durch Einführung und Vorstellung, Interaktion, Rückfragen und Abschluss belegt. Je interaktiver Sie die eLecture halten, desto besser: besprechen Sie sich mit Ihrer CoModeration!
%	\end{block}
%\end{frame}

{\setbeamercolor{background canvas}{bg=digiPH_orange!20}
\begin{frame}
	%\maketitle
	% \maketitle funktioniert auch im Article-Modus
	 \titlepage
\end{frame}
}
\end{withoutheadline}

\miniframesoff
\begin{frame}[label=inhalt]{Overview}
	\tableofcontents
\end{frame}
\miniframeson

\section{Pendahuluan}

\begin{frame}{LATAR BELAKANG}
Kawasan waterfront memiliki karakteristik dan perhatian khusus mengingat pentingnya air sebagai sumber kehidupan. Untuk mencapai tujuan tersebut, pengembangan tepi laut adalah sangat penting. Menurut \cite{hussein2014}, pengembangan tepi laut yang baik adalah yang mempertimbangkan keberagaman, interaksi komunitas, kenyamanan dan keamanan, lingkungan dan keberlanjutan.


%	\begin{block}{Zur Didaktik: Ablauf und Lernziele zu Beginn sind sehr wichtig!\\
%		\small Folgendes gilt aber auch für alle anderen Auflistungen:}
%		\end{block}
\end{frame}

\begin{frame}
Menurut \cite{breen1994}, tekanan pada ruang kota dan infrastruktur, kebutuhan atas kualitas lingkungan, dan ketersediaan ruang tepi laut yang terbengkalai menjadi alasan pengembangan ulang kawasan tepi laut sebagai solusi yang pas.

\end{frame}

\begin{frame}[label=current]{Kota Parepare}
Kota Parepare merupakan kota yang terletak di Provinsi Sulawesi Selatan. Peningkatan jumlah penduduk di Parepare berkisar 2\%, pada tahun 2019 Parepare memiliki penduduk sebanyak 145.178 orang \citep{bpskotaparepare2020}.

Saat ini, Kota Parepare sedang melakukan sejumlah kemajuan di bidang pariwisata. Salah satunya adalah revitalisasi tepi laut senggol.
\end{frame}

\begin{frame}[label=current]{Lokasi Penelitian} \vspace{4pt}
 	\begin{center}
		\includegraphics[width=\textwidth]{figures/lokzi.jpg}
		{\tiny \textcolor{digiPH_darkorange}{Sumber: Penulis, \href{https://creativecommons.org/licenses/by/3.0/at/}{CC BY}}}
	\end{center}
\end{frame}

\begin{frame}[label=current]
Kawasan tepi laut senggol terbentang dari Pelabuhan Nusantara hingga Pasar Senggol sepanjang sekitar 300 meter. Sepanjang garis pantai tersebut terbentuk sejumlah ruang dengan karakteristik yang berbeda.
\end{frame}
\begin{frame}[label=current]{Preferensi} \vspace{4pt}
Menurut \cite{devysandra2012}, preferensi adalah kecenderungan untuk memilih sesuatu yang lebih disukai daripada yang lain. Sejumlah atribut pada ruang tersebut menjadi alasan dalam pemilihan ruang di kawasan tepi laut.

Mengetahui preferensi ruang dari masyarakat dapat membantu menyediakan dan mengelola pengembangan untuk memenuhi kebutuhan masyarakat secara efektif \citep{madureira2018}.

\end{frame}
\begin{frame}[label=current] \vspace{4pt}

Dengan beragam ciri khusus masyarakat dan latar belakang yang berbeda, tepi laut senggol diharapkan dapat dikaji agar memenuhi kebutuhan masyarakat lokal dan pengunjung yang transit dari berbagai daerah di Sulawesi Selatan.
Selain memperhatikan preferensi ruang masyarakat, isu-isu berkaitan dengan keberlanjutan akan di selediki dalam mendukung tepi laut sukses berkelanjutan.
\end{frame}


\begin{frame}[label=current]{RUMUSAN MASALAH} \vspace{4pt}

Pada tahun 2011, kota Parepare memulai perencanaan penataan kawasan tepi laut senggol. Penataan ini menghasilkan sejumlah ruang yang memiliki atribut yang berbeda. Saat ini, masyarakat terpecah dalam menggunakan ruang di kawasan waterfront. Ada masyarakat yang cenderung terhadap ruang satu daripada lainnya. Alasan pemilihan ini belum jelas, seperti \cite{campagnaro2020} menemukan elemen buatan seperti jalan setapak, kursi, kran air minum berperan penting dalam pemilihan ruang hijau. Teori tersebut perlu dikaji pada kawasan ini, dimana elemen buatan tampaknya tidak berpengaruh signifikan terhadap pemilihan ruang di kawasan tepi laut.
\end{frame}

\begin{frame}[label=current]{Pertanyaan Penelitian} \vspace{4pt}
\begin{itemize}
\item Apa preferensi ruang masyarakat di kawasan tepi laut Senggol? Mengapa masyarakat memilih satu daripada lainnya?
\item Apakah fitur ruang adalah faktor terpenting untuk preferensi ruang masyarakat? Apakah kepentingannya bervariasi diantara ruang-ruang?
\end{itemize}
\end{frame}

\section{Tinjauan Pustaka}
\begin{frame}[label=current]{Fitur Ruang} \vspace{4pt}
Preferensi ruang masyarakat kemungkinan besar didasari oleh faktor perseptual dan estetika \citep{vandenberg2003,england2009}.

\cite{england2009} memaparkan kualitas perseptual dapat berbeda tergantung pada tipe ruang publik yang digunakan.
Hal tersebut menyangkut pada keberagaman, kontras dan warna serta kemungkinan kecil keberadaan atau jumlah dari sebuah fitur individu \citep{swanwick2009}. \cite{wang2021} merumuskan fitur ruang adalah kebisingan, fasilitas, keamanan, estetika, dan pemeliharaan.
\end{frame}

\begin{frame}[label=current]{Socio-demografi} \vspace{4pt}
Socio-demografi memiliki asosiasi terhadap frekuensi penggunaan suatu taman kota \citep{azagew2020}. Variabel-variabel sosio-demografi mungkin mempengaruhi preferensi ruang\textit{(spatial preference)}\citep{zhao2020}.

\end{frame}
\begin{frame}[label=current]{Aspek Socio-demografi} \vspace{4pt}
\begin{itemize}
\item Gender
    \item Umur
    \item Pekerjaan/Jumlah pendapatan
    \item Ras
\end{itemize}
\end{frame}





\begin{comment}
\begin{frame}{Lorem Ipsum}
	\begin{center}
		\includegraphics[height=5.5cm]{figures/placeholder}

		{\tiny \textcolor{digiPH_darkorange}{\lipsum[1][1] \url{penulis}}}
	\end{center}
	\begin{center}
		\textbf{Lorem Lipsum} \lipsum[1][1]

	\end{center}
\end{frame}

\begin{frame}
	\begin{center}
		\includegraphics[height=5.5cm]{figures/placeholder}
		{\tiny \textcolor{digiPH_darkorange}{\lipsum[1][1] \url{penulis}}}
	\end{center}
	\begin{center}
		{\small\textbf{Lorem Lipsum} \lipsum[1][1]}
	\end{center}
\end{frame}


\section{Lorem}


\begin{frame}{Lorem Ipsum}
	\small \lipsum[1][1-2]
%	\tiny Weiteres dazu entnehmen Sie bitte dem Infoblatt: \url{http://www.virtuelle-ph.at/selbst-electures-abhalten/infoblatt}

%	\small Die Teilnehmenden werden über einen Disclaimer zu Beginn der eLecture darüber informiert:

	\begin{center}
		\includegraphics[height=5cm]{figures/placeholder}

		{\tiny \textcolor{digiPH_darkorange}{Sumber: Penulis, \href{https://creativecommons.org/licenses/by/3.0/at/}{CC BY}}}
	\end{center}
\end{frame}

%	\tiny Weiteres dazu entnehmen Sie bitte dem Infoblatt: \url{http://www.virtuelle-ph.at/selbst-electures-abhalten/infoblatt}

%	\small Die Teilnehmenden werden über einen Disclaimer zu Beginn der eLecture darüber informiert:

%-------------------------------------------------------------------------------------
%\begin{comment}
\section{Lorem}

\begin{frame}{Highlight}
	\begin{center}
		\Large
		online immer lieber {\Huge mehr} als weniger\\
		und häufig wechseln!
	\end{center}
	\pause
	\begin{center}
		\Large
		Das bringt Abwechslung und Bewegung\\auf
        den Bildschirm und hilft,\\
        {\Huge Aufmerksamkeit} zu {\Huge halten}.
	\end{center}
\end{frame}

\begin{frame}{Kolom gambar dan teks}
	\begin{columns}
		\begin{column}{5cm}
			\begin{center}
				\includegraphics[height=6cm]{figures/chamaeleon_hochformat}

				{\tiny \textcolor{digiPH_darkorange}{Bildquelle: \url{pixabay.com}, CC-0}}
			\end{center}
		\end{column}
		\begin{column}{5cm}
			\begin{center}
				\Large
				Bitte mit Bedacht!
			\end{center}
			\pause
			\begin{center}
				Nicht mehr als ein bis zwei verschiedene Schriftfarben verwenden (Ruhe, bessere Lesbarkeit).
			\end{center}
			\begin{center}
				Tipp: Lieber durch Bilder Abwechslung schaffen!
			\end{center}
		\end{column}
	\end{columns}
\end{frame}

\begin{frame}
	%Gambar besar dengan penjelasan tanpa judul
	\begin{center}
		\includegraphics[height=6cm]{figures/kopfhoerer}

		{\tiny \textcolor{digiPH_darkorange}{Nicht die Bildquelle und die Lizenz vergessen! Bildquelle: \url{pixabay.com}, CC-0}}
	\end{center}
	\begin{center}
		Haben Sie Mut zur Vereinfachung und zu ungewohnter Positionierung und finden Sie interessante Ausschnitte.
	\end{center}
\end{frame}

\begin{frame}{Gambar besar dengan judul dan penjelasan}
	\begin{center}
		\includegraphics[height=5.5cm]{figures/placeholder}

		{\tiny \textcolor{digiPH_darkorange}{Bildquelle: \url{pixabay.com}, CC-0}}
	\end{center}
	\begin{center}
		Geschlechtsneutrales Formulieren ist uns wichtig: bei unseren eLectures ebenso. \textbf{In Schrift, Wort und Bild!}
	\end{center}
\end{frame}


\begin{frame}{Gambar dengan penjelasan sebelumnya}
	\small Ihre eLecture wird zum Nachsehen aufgezeichnet und veröffentlicht. Deshalb muss sie inhaltlich und urheberrechtlich korrekt sein!

	\tiny Weiteres dazu entnehmen Sie bitte dem Infoblatt: \url{http://www.virtuelle-ph.at/selbst-electures-abhalten/infoblatt}

	\small Die Teilnehmenden werden über einen Disclaimer zu Beginn der eLecture darüber informiert:

	\begin{center}
		\includegraphics[height=3.5cm]{figures/placeholder}

		{\tiny \textcolor{digiPH_darkorange}{Bildquelle: Lene Kieberl, \href{https://creativecommons.org/licenses/by/3.0/at/}{CC BY}}}
	\end{center}
\end{frame}

\begin{frame}[t]{Tiga Gambar}

\hfil\hfil\includegraphics[width=5cm]{figures/placeholder}\newline
  \null\hfil\hfil\makebox[5cm]{Lorem}\newline
  \vfil
  \hfil\hfil{\includegraphics[width=5cm]{figures/placeholder}}\hfil\hfil
    {\includegraphics[width=5cm]{figures/placeholder}}\newline
  \null\hfil\hfil{\makebox[5cm]{Lorem}}
    \hfil\hfil{\makebox[5cm]{Lorem}}

\end{frame}

\begin{frame}{Kolom dengan foto orang}
	\begin{columns}
		\begin{column}{5cm}
			\small besonders aber bei Minderjährigen bitte unbedingt abklären, ob diese bzw. ihre Erziehungsberechtigten mit der Veröffentlichung einverstanden sind (Recht am eigenen Bild).
		\end{column}
		\begin{column}{5cm}
			\begin{center}
				\includegraphics[height=2.5cm]{figures/placeholder}

				{\tiny \textcolor{digiPH_darkorange}{Bildquelle: Details Foto XY, \url{pixabay.com}, CC-0}}
			\end{center}
		\end{column}
	\end{columns}
	\begin{columns}
		\begin{column}{5cm}
			\begin{center}
				\includegraphics[width=5cm]{figures/placeholder}

				{\tiny \textcolor{digiPH_darkorange}{Bildquelle: Details Foto XY, \url{pixabay.com}, CC-0}}
			\end{center}
		\end{column}
		\begin{column}{5cm}
			\begin{block}{\small Tipp:}
				\scriptsize
				Als Hilfestellung können Sie vielleicht unseren \enquote{Schummelzettel OER} mit Tipps und Quellen für verwendbare Bilder nutzen, den Sie hier zum Download vorfinden:\\
				\url{http://www.virtuelle-ph.at/schummelzettel}
			\end{block}
		\end{column}
	\end{columns}
\end{frame}

\begin{frame}{Sie möchten Software/Seiten live vorzeigen?}
	% Schriftgröße verkleinern
	\scriptsize
	% Absatzabstand einstellen
	\setlength{\parskip}{0.75\baselineskip}

	Beachten Sie bitte, dass Sie beim Zeigen von längeren Ausschnitten einer \textbf{Programmoberfläche} (Vorzeigen von Abläufen die nicht mehr durch das Zitatrecht abgedeckt sind) gegebenenfalls die \textbf{Erlaubnis der Herstellerfirma einholen müssen}, sofern Sie nicht Urheber/in sind bzw. keine lizenzrechtliche Erlaubnis zur Vorführung vorliegt (mehr Info: \url{http://bit.ly/vphuhr}).

	\textbf{Diese Angabe kann Ihre Co-Moderation für Sie einblenden, bitte teilen Sie sie bei Testtermin oder via Email mit.}

	\begin{block}{\scriptsize Beispiel: Live-Demo \href{GIMP}{http://www.gimp.org/},GNU General Public License 29.01.2018}
		\centering
		\includegraphics[height=4cm]{figures/placeholder}
	\end{block}
\end{frame}

%\end{comment}
%\begin{comment}

% Hanya untuk Navigasi Miniframe
\miniframesoff
\begin{frame}<beamer>{}
	\begin{center}
		\ldots Manege frei für Ihr Knowhow und Ihre Folien!
	\end{center}
	\vspace{3cm}
	\tiny
	PS: Am Ende Ihrer eLecture weist Ihre Co-Moderation die Teilnehmenden noch auf weitere Angebote (\zB der aktuellen Online-Tagung) und die Social-Media-Präsenzen hin. Wir bitten Sie auch um Weiterverbreitung unseres (und Ihres!) Lernangebots. Selbst schon geliked? Wir freuen uns über Likes, Kommentare und natürlich Follower!
\end{frame}
%\end{comment}

%\begin{frame}{\enquote{Stopfen} Gambar dengan qoute digambar \ldots}
%	\framesubtitle{Arbeiten Sie mit Bildern: online lieber mehrere, weniger dichte Folien zeigen!}
%	\begin{center}
%		\begin{tikzpicture}
%			\node[anchor=south west,inner sep=0] (bild) at (0,0) {\includegraphics[width=0.88\textwidth]{figures/placeholder}};
%			\begin{scope}[x=(bild.south east),y=(bild.north west)]
%				% Gitternetz einzeichnen
%				%\draw[help lines] (0,0) grid[xstep=0.1,ystep=0.1] (1,1);
%				% Text einfügen
%				\node[font=\scriptsize,align=left,color=white,text width=4cm] (text) at (0.79,0.3) {Bitte verwenden Sie ausschließlich urheberrechts-einwandfreies Bildmaterial (NoGo: Schulbücher!)\\
%				Am besten also: eigenes und/oder CC-Bildmaterial, \zB von \url{www.pixabay.com} oder ähnlichen Plattformen!\\
%				Bitte nie eine sichtbare Bildquelle und Lizenz vergessen.};
%			\end{scope}
%		\end{tikzpicture}
%
%		{\tiny \textcolor{digiPH_darkorange}{Bildquelle: David Bogner, 2015, \href{https://creativecommons.org/licenses/by-nc-sa/3.0/at/}{CC BY-NC-SA}}}
%	\end{center}
%\end{frame}

\end{comment}
\begin{frame}<beamer>{}
\bibliographystyle{apalike}
{\tiny
\bibliography{biblio.bib}
}
\end{frame}

\end{document}
